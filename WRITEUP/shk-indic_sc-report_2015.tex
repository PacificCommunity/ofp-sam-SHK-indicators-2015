%%%%%%%%%%%%%%%%%%%%%%%%%%%%%%%%%%%%%%%%%
% Journal Article
% SHARK INDICATORS IN THE WESTERN CENTRAL PACIFIC
% June 2015
%
% Author: Joel Rice (joelrice@uw.edu )
%
%
%%%%%%%%%%%%%%%%%%%%%%%%%%%%%%%%%%%%%%%%%

%----------------------------------------------------------------------------------------
%  PACKAGES AND OTHER DOCUMENT CONFIGURATIONS
%----------------------------------------------------------------------------------------



\documentclass[12pt]{article}

\input{input_package.tex}


\graphicspath{ {C:/Users/sheltonh/Dropbox/SHK-indicators-2015/GRAPHICS/} }
\usepackage[english]{babel}

\input{WCPFC-title-page.tex}
\reportauthor{Joel Rice\footnote{Joel Rice Consulting Ltd.}, Laura Tremblay-Boyer, and Shelton Harley}
\reporttitle{Analysis of stock status and related indicators for key shark species of the Western Central Pacific Fisheries Commission}
\reportnumber{SA-IP-05}



%----------------------------------------------------------------------------------------
%  Document content
%----------------------------------------------------------------------------------------


\begin{document}

\wcpfctitlepage

\section*{Executive Summary}


\section{Introduction}
The status of the many  shark species, espically those designated as  (\emph{key shark species}) in the western and central Pacific Ocean was is under review and instead of doing a southern blue shark stock assessment you all asked for this. An indicator analysis of blue, mako, thresher, silky and oceanic white tip sharks in the waters of the WCPO. 
  We didn't do any fancy assessment work or models, but rather make colorul plots and tabulate laregly useless statistics ( the geometric mean of the sandardized counts of other sharks has decreased relative to the base year but is comparable to the initial year).  All in all this paper should give you a good understanding of the uncertainty regarding any species population viability, and even more confusioin regarding what we can, have or would do about it.  Becauses sharks are often caught as bycatch in the Pacific tuna fisheries (though some directed mixed species fisheries, sharks and tunas/billfish, do exist) sharks are doomed.
  
  While we cannot specify a percent  reduction in fishing mortality  of approximately  needed for any specific  species to reach  MSY levels in the western central Pacific Ocean, we do know that-  based on modeling of the factors influenciing the catch rate - the most effective way to improve population outlook   would be the banning of shark lines.



\section{General Methods}
        \subsection{Description of Data Sources}
         
\addcenterfig[Map of WCPO and regions used for the analysis.]{fig:regions}{FIG_1_MAP}
\addcenterfig[Map of WCPO and observed effort and observed shark catch. ]{fig:sets}{FIG_2_MAP_sets}
\clearpage         
        \subsection{Data formatting}
        \subsection{Limitations - Caveats}
        
        
%----------------------------------------------------------------------------------------
%  Distribuion Indicators
%----------------------------------------------------------------------------------------
             
        
\section{Distribution Indicator Analyses}
      \subsection{Introduction}
      \subsection{Methods}
      \subsection{Results}
          \subsubsection{Fishing Effort}
              
\addcenterfig[Aggregate effort by region. \hl{needs updating}]{fig:aggeff}{FIG_xx_agg_eff}
\addcenterfig[Observed effort by region.]{fig:obseff}{FIG_xx_OBS_eff}
\addcenterfig[Logsheet effort by month.]{fig:logeffmnth}{FIG_xx_LOGSHEET_mm}
\addcenterfig[Observed effort by month.]{fig:obseffmnth}{FIG_xx_obsBY_mm}
\addcenterfig[Absolute percent difference in effort between reported (logsheet)  effort and observed effort.]{fig:effdiff}{FIG_xx_obsDIFFlog_mm}
\clearpage         
%----------------------------------------------------------------------------------------
          \subsubsection{Blue Shark}

\addcenterfig[Blue shark distribution indicators. Proportion of positive sets, observer data.]{fig:bsh1}{FIG_xx_pcntpos_reg_BSH}
\addcenterfig[Blue shark distribution indicators. Proportion of 5 degree squares having CPUE greater than 1 per 1000 hooks region, observer data.]{fig:bsh2}{FIG_xx_HIGH_CPUE_BSH}


%----------------------------------------------------------------------------------------
\subsubsection{Mako Shark}
          
\addcenterfig[Mako shark distribution indicators. Proportion of 5 degree squares having CPUE greater than 1 per 1000 hooks region, observer data.]{fig:mak2}{FIG_xx_HIGH_CPUE_MAK}
          
          
%----------------------------------------------------------------------------------------
 \subsubsection{Silky Shark}
          
\addcenterfig[Silky shark distribution indicators. Proportion of 5 degree squares having CPUE greater than 1 per 1000 hooks region, observer data.]{fig:fal2}{FIG_xx_HIGH_CPUE_FAL}

%----------------------------------------------------------------------------------------
 \subsubsection{Oceanic Whitetip Shark}
          
\addcenterfig[Oceanic whitetip shark distribution indicators. Proportion of 5 degree squares having CPUE greater than 1 per 1000 hooks region, observer data.]{fig:ocs2}{FIG_xx_HIGH_CPUE_OCS}

%----------------------------------------------------------------------------------------
  \subsubsection{Thresher Shark}

\addcenterfig[Thresher shark distribution indicators. Proportion of 5 degree squares having CPUE greater than 1 per 1000 hooks region, observer data.]{fig:thr2}{FIG_xx_HIGH_CPUE_THR}
          
          
      \subsection{Conclusions}
%----------------------------------------------------------------------------------------
%  Species COmposition
%----------------------------------------------------------------------------------------
      
      
\section{Species Composition Indicator Analyses}
      \subsection{Introduction}
      \subsection{Methods}
      \subsection{Results}
      
      
\addcenterfig[Catch Composition Indicators. Sharks Per. 1000 hooks by region, observer data.]{fig:catchcomp}{FIG_xx_shksP1000Hooks}
          
      
      \subsection{Conclusions}

%----------------------------------------------------------------------------------------
%  CPUE INDICATORS
%----------------------------------------------------------------------------------------
\section{Catch Per Unit Effort indicator analyses}
      \subsection{Introduction}
      \subsection{Methods}
      \subsection{Results}
      %----------------------------------------------------------------------------------------
         \clearpage
         \subsubsection{Blue Shark}
           
\addcenterfig[Blue shark CPUE indicators. Proportion of positive sets, observer data.]{fig:bshcp1}{FIG_xx_pcntpos_reg_BSH}
\addcenterfig[Blue shark CPUE indicators. Nominal CPUE, sharks per 1000 hooks, observer data.]{fig:bshcp2}{FIG_xx_nomCPUE_reg_BSH}

%----------------------------------------------------------------------------------------
  \subsubsection{Mako Shark}
          
\addcenterfig[Mako shark CPUE indicators. Proportion of positive sets, observer data.]{fig:makcp1}{FIG_xx_pcntpos_reg_MAK}
\addcenterfig[Mako shark CPUE indicators. Nominal CPUE, sharks per 1000 hooks, observer data.]{fig:makcp2}{FIG_xx_nomCPUE_reg_MAK}
\clearpage
%----------------------------------------------------------------------------------------
 \subsubsection{Silky Shark}
 
\addcenterfig[Silky shark CPUE indicators. Proportion of positive sets, observer data.]{fig:falcp1}{FIG_xx_pcntpos_reg_FAL}
\addcenterfig[Silky shark CPUE indicators. Nominal CPUE, sharks per 1000 hooks, observer data.]{fig:falcp2}{FIG_xx_nomCPUE_reg_FAL}

%----------------------------------------------------------------------------------------
 \subsubsection{Oceanic Whitetip Shark}
          
\addcenterfig[Oceanic whitetip shark CPUE indicators. Proportion of positive sets, observer data.]{fig:ocscp1}{FIG_xx_pcntpos_reg_OCS}
\addcenterfig[Oceanic whitetip shark CPUE indicators. Nominal CPUE, sharks per 1000 hooks, observer data.]{fig:ocscp2}{FIG_xx_nomCPUE_reg_OCS}

%----------------------------------------------------------------------------------------
  \subsubsection{Thresher Shark}
          
\addcenterfig[Thresher shark CPUE indicators. Proportion of positive sets, observer data.]{fig:thrcp1}{FIG_xx_pcntpos_reg_THR}
\addcenterfig[Thresher shark CPUE indicators. Nominal CPUE, sharks per 1000 hooks, observer data.]{fig:thrcp2}{FIG_xx_nomCPUE_reg_THR}
          
%----------------------------------------------------------------------------------------          
      \subsection{Conclusions}            
      
 \clearpage     
      
      
      
      
     
%----------------------------------------------------------------------------------------
%  Biological
%----------------------------------------------------------------------------------------
      
\section{Biological indicator analyses}
      \subsection{Introduction}
      \subsection{Methods}
      \subsection{Results}
      \subsection{Conclusions}
 \clearpage     
      
%----------------------------------------------------------------------------------------
%  Discussion Type Sections.........
%----------------------------------------------------------------------------------------
         
      
\section{Feasibility of Stock Assessments}
\section{Impact of Recent Shark Management Measures}
\section{Recommendations for Future Indicator Work}
\section{Management Implications}

\section*{Acknowledgements}

%----------------------------------------------------------------------------------------
%  REFERENCE LIST
%----------------------------------------------------------------------------------------



 \section{Appendices}
% 


      
      
\end{document}




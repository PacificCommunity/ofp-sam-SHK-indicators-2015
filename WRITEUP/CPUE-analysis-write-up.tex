%\documentclass{SCreport}

%\begin{document}
%%%%%%%%% Defining latex macros to print tables %%%%%%%%%%


\newcommand{\blueaic}{
% latex table generated in R 3.2.0 by xtable 1.7-4 package
% Sat Jul 04 14:52:03 2015
\begin{table}[ht]
\centering
\caption{AIC improvement over null model for blue from a single explanatory variable} 
\label{blue:aic1}
\begin{tabular}{lrr}
  \hline
Variable & AIC.diff & AIC.diff.sigma \\ 
  \hline
program\_code & 24382.47 & 16607.68 \\ 
  flag\_id & 23711.29 & 17163.94 \\ 
  HPBCAT2 & 20183.28 & 5765.59 \\ 
  HPBCAT & 20178.74 & 5576.42 \\ 
  yy & 14357.75 & 8612.54 \\ 
  mm & 3725.06 & 2511.55 \\ 
  sharkbait & 994.97 & 454.07 \\ 
   \hline
\end{tabular}
\end{table}

}


\newcommand{\makonorthaic}{
% latex table generated in R 3.2.0 by xtable 1.7-4 package
% Sat Jul 04 14:52:03 2015
\begin{table}[ht]
\centering
\caption{AIC improvement over null model for mako.north from a single explanatory variable} 
\label{mako.north:aic1}
\begin{tabular}{lrr}
  \hline
Variable & AIC.diff & AIC.diff.sigma \\ 
  \hline
HPBCAT2 & 6433.14 & 642.33 \\ 
  HPBCAT & 6417.24 & 640.61 \\ 
  mm & 2845.80 & 216.50 \\ 
  yy & 2216.05 & 401.91 \\ 
  flag\_id & 115.74 & 61.51 \\ 
  program\_code & 114.17 & 44.28 \\ 
  sharkbait & 15.46 & 3.29 \\ 
   \hline
\end{tabular}
\end{table}

}


\newcommand{\makosouthaic}{
% latex table generated in R 3.2.0 by xtable 1.7-4 package
% Sat Jul 04 14:52:03 2015
\begin{table}[ht]
\centering
\caption{AIC improvement over null model for mako.south from a single explanatory variable} 
\label{mako.south:aic1}
\begin{tabular}{lrl}
  \hline
Variable & AIC.diff & AIC.diff.sigma \\ 
  \hline
program\_code & 1693.97 & 251.6 \\ 
  flag\_id & 1438.39 & --- \\ 
  HPBCAT2 & 1202.23 & 79.45 \\ 
  HPBCAT & 1127.42 & 34.67 \\ 
  yy & 875.66 & 389.5 \\ 
  mm & 733.11 & 178.28 \\ 
  sharkbait & 8.37 & -1.89 \\ 
   \hline
\end{tabular}
\end{table}

}


\newcommand{\ocsaic}{
% latex table generated in R 3.2.0 by xtable 1.7-4 package
% Sat Jul 04 14:52:03 2015
\begin{table}[ht]
\centering
\caption{AIC improvement over null model for ocs from a single explanatory variable} 
\label{ocs:aic1}
\begin{tabular}{lrr}
  \hline
Variable & AIC.diff & AIC.diff.sigma \\ 
  \hline
yy & 3726.21 & 958.91 \\ 
  program\_code & 2781.62 & 1369.09 \\ 
  flag\_id & 1499.89 & 772.13 \\ 
  sharkbait & 887.50 & 103.91 \\ 
  HPBCAT2 & 308.22 & 1656.57 \\ 
  HPBCAT & 256.09 & 1658.57 \\ 
  mm & 143.92 & 185.56 \\ 
   \hline
\end{tabular}
\end{table}

}


\newcommand{\silkyaic}{
% latex table generated in R 3.2.0 by xtable 1.7-4 package
% Sat Jul 04 14:52:03 2015
\begin{table}[ht]
\centering
\caption{AIC improvement over null model for silky from a single explanatory variable} 
\label{silky:aic1}
\begin{tabular}{lrr}
  \hline
Variable & AIC.diff & AIC.diff.sigma \\ 
  \hline
program\_code & 10913.31 & 6886.75 \\ 
  flag\_id & 10144.24 & 5446.47 \\ 
  sharkbait & 5382.46 & 1713.24 \\ 
  yy & 4986.96 & 3273.31 \\ 
  HPBCAT2 & 1639.75 & 4527.65 \\ 
  HPBCAT & 1628.44 & 4513.75 \\ 
  mm & 957.50 & 358.96 \\ 
   \hline
\end{tabular}
\end{table}

}



\subsection{Introduction}
\hlgreen{This text could be smoothed over, paragraphs 3-5 need to be merged.}
%This paper follows from the previous indicator based analysis presented to the Western and Central Pacific Fisheries Commission (WCPFC) Scientific Committee (SC7, Clarke et al. 2011), stock assements (Rice et al. 2014, Rice et al.2013, Rice et al.2012) \hllb{(cite the standardization papers, cite ISC work?)}. The developments presented here include additional analyses of the Secretariat of the Pacific (SPC) data holdings for silky caught in longline and purse seine fisheries in the Western and Central Pacific Ocean (WCPO), though we note that some previous data (Japan) was not available for this effort. Standardized catch per unit of effort (CPUE) series are developed for the main shark species.  
\emph{Note: The current analysis does not construct inputs to use for stock assessments or catch estimates. Our goal is to highlight general trends in population abundance over time, to be interpreted together with other indicators as outlined above. We recommend that catch rates standardization for stock assessments or catch estimates be conducted independently.}


% intro      
Catch-per-unit-effort data (CPUE) are commonly used as an index of abundance for marine species. However, multiple factors---fishing technique, season, bait type, etc.---can alter the relationship between CPUE and abundance, especially in complex fisheries systems comprising of several fleets and spanning large spatial and temporal scales. Nominal catch rates must thus be standardized to account for changes in the relative prevalence of these factors over time. This is typically done \via the use of models in the GLM family, which allows us to model the relationship of CPUE \vs a set of explanatory variables to be standardized against, but these variables must be defined for each observation. The dataset used in the current analysis provides many such candidate variables, but, given the diversity of observer programs represented, few had enough coverage to be retained in the final models. The available variables are described in Table \ref{tbl:glm-vars} (see also Table 2 in \citealt{Francis2014_a} for an overview of the use of variables in shark CPUE standardizations).

% issue with sharks cpue standrz
CPUE data for species such as sharks often have a large proportion of observations (sets) with zero catch, while at the same time also including instances of large catches (`long tails'). These uncommon instances of high catches can occur when areas of high shark densities are accidentally encountered, but also when fishing vessels engage in anecdotal shark targeting behaviour. The co-existence of both high proportions of zero catches \emph{and} long tails resuls in over-dispersed data, and is typical of bycatch species \citep{Ward2005_a}. These features are challenging to account for from a statistical point of view, and have been reviewed at length in the literature on by-catch analyses (Bigelow et al.  2002; Campbell  2004, Ward and Myers  2005; Minami et al. 2007).

Zero-inflated approaches have been advocated as best suited to model overdispersion in both zeros and positive counts \citep{Brodziak_2013_a}. A drawback of the zero inflated approach is that it is data intensive and that models fail to converge. In addition, in practice it tends to be more successful at modelling the extra zeroes than the large-catch events, since the dispersion parameter which controls the length of the tail is assumed to be constant over all factors. This is especially a problem when the mean of the distribution is close to zero or one, as in those instances the probability of getting large events if mostly controlled by the dispersion parameter $\sigma$, unlike when the mean is larger and the tail is not as pronounced. However, whenever conditions are good for sharks and/or targeting takes place, larger catch events can happen and not modelling those properly implies that we are missing important drivers behind catch rates. Typically, this can be diagnosed as a departure from the one-to-one line on the right-hand side of quantile-quantile plots. Here we achieved significant improvements in model diagnostics by keeping a negative binomial distribution but allowing both the mean $\mu$ and $\sigma$ parameters to be modelled against covariates \citep{GAMLSSXXXX}. This approach is much less computationally intensive than zero-inflated models, and yielded excellent diagnostics with the current dataset.


 \subsection{Methods}
 \hlgreen{Review: high priority text is rough}\hlcoral{Tables to be moved to the end of the document}
 \label{cpuemeth:datafilter}
 \subsubsection{Stock definition for the purpose of the analysis}
   
% talk about stocks and year span 
Standardized CPUE series for the longline bycatch fisheries were developed using generalized linear models. We did not generate indices from purse-seine observer data. The number of hooks in a longline set was used as a measure of effort.
See also Clarke et al., (2011, 2011b),  Walsh and Clarke (2011), Rice and Harley (2013) (Rice et al. 2014, Rice et al.2013, Rice et al.2012)  for past work on shark CPUE standardization in the Western Central Pacific.
                                                                                       
%Silky and oceanic white tip sharks are observed mainly in the equatorial waters in the purse seine fishery (Figure \ref{...}), and from about -25\degree~S to 25\degree~N in the longline fishery (Figure 1). 
Silky and oceanic white tip sharks have been assessed \citep{Rice2012_a, Rice2013_a} as a single stock in the WCPO, and are presented in this analysis as a single stock.  Thresher, mako and blue sharks occur more frequently in cold, temperate waters, and generally believed to be separated into northern and southern stocks. For instance, blue sharks in the North Pacific have been subject to multiple stock assessments as a single stock \citep{XXXX}. These temperate species will thus be analysed as individual stocks. Porbeagle sharks are only found in the southern hemisphere and will also be analysed as a single stock. Hammerhead sharks, for the purpose of this analysis (and throughout the report) are analyzed as one stock.

\subsubsection{Data Preparation}
To further define the expected geographic range, we defined a coarse climate `envelope' based on sea surface temperature. This aided in distinguishing between zero catches in areas where the species does not occur from zero catches in areas where the species occurs but was not caught. Temperature data were downloaded from the GODAS database \citep{GODASXXXX_a} and matched to the observer data on a set by set basis. The temperature range of a species was defined as the minimum and maximum of the monthly mean sea surface temperatures (SST) of cells with positive catches for that species (see Table \ref{meth:temprange}). The temperature predicted by GODAS at the 5 meters depth at settime/day was used for the SST value. Only cells for which all mean monthly temperatures fell within this range were retained. %\hllb{REWORK. Note that this is a minimal filter and it was only used to exclude the most improbable cells from which a species could be seen.}
 %\hl{Looked at range of SST where positive catches occured, selected cells where median falls within this range.}
                                                                                       
Data were cleaned following the general method outlined in Appendix section \ref{app:datacleaning}. Records from the US observer programs (Hawaii and American Samoa) were excluded from the analysis as they were only available up to 2011. Records from any observer programs for which we had less than 100 sets were removed. Extreme catch events greater than the $97.5^{th}$ quantile of the nominal CPUE were also removed. Finally, year effects were only estimated if there were at least 50 sets observed in that year (by sepcies and population).  Records from the Papua New Guinea observer program were removed as vessels in the fleet frequently target sharks.
                                                                                      
Previous work \citep{Rice2012_a, Rice2013_a, Bromhead et al 1012} has examined the extent of shark targeting in the WCPO tuna longline fleet, and found that a number of factors (i.e. sharklines, wire trace can lead to a higher catch rate). Shark target sets comprise a much smaller proportion of the overall dataset (6.5\% of the sets), however these targeting sets represent significant shark catch (82\% of the total silky shark catch). Therefore the dataset was examined with respect to variables relating to whether sharks were the intentional target of the set.                                                                                      
Shark targeting sets were deemed to be sets where the observer had marked that the set was intentionally targeting sharks of any species, \hlam{whether shark bait was used, or whether shark lines were used}. % We also removed the data from the PGOB because of frequent shark targeting. - edit said two sentences ago
Details of the  filtering rules including the number of records filtered and the filtering  order    are listed by species \hlcoral{in Table} \ref{XXX}. 
                                                                                       
% \subsection{Additional categorical variables}    % jr edits don't think we need that
% 1. Day category: sunset and sunrise were calculated\\ 
% 2. HPBCAT: based on data exploration two types explored: either shallow or deep (with shallow $<=$ 10), or HPBCAT2 shallow, intermediate and deep, with intermediate between 10 and 15 ... in general hpbcat2 had slighlty better aic %(in tables at the end)
     %Latitude and longitude were truncated to the nearest 1\degree; this location information was used to calculate the set specific association with a 5\degree square (referred to hereinafter as cell). Date of set was used to calculate the year, month, quarter and trimester of the set. Set time was used to calculate the time category of the day in sixths starting at midnight. A non-target data set was created as a result of filtering data sets according to the above rules as well as filtering sets where sharks were the intentional target. This was done under the premise that the factors leading to non-zero catch rates when targeting sharks would be different than factors that lead to non-zero catch rates when not targeting sharks. 
                                                                                       
                                                                                       
   \subsection{Overview of GLM Analyses}
   \subsubsection{Notes on error distributions:} 
   \hllb{\# some repeat from intro, gah}
   The  filtered datasets  were standardized using generalized linear models (McCullagh and Nelder 1989) using the software package R (\url{www.r-project.org}). Initially multiple assumed error structures were tested including the delta lognormal approach (DLN) (Lo et al. 1992, Dick 2006, Stefansson 1996, Hoyle and Maunder 2006), zero-inflated poisson and negative binomial models, the tweedie distribution \citep{Shono2008},%  in JR_bib_edts.bak Hiroshi Shono 2008. Application of the Tweedie distribution to zero-catch data in CPUE analysis. Fisheries Research 93 (2008) 154-162
   and negative binomial models with mu and  $\sigma$ modelled. Due to its superior performance both in run time and model diagnostics, we retained the latter and only present those results here.

The negative binomial (Lawless 1987) is typically more robust to issues of overdispersion (overdispersion can arise due to excess zeros, clustering of observations, or from correlations between observations) was also used. This model has been advocated as a model that is more robust to overdispersion than the Poisson distribution (McCullagh and Nelder 1991), and is appropriate for count data (Ward and Myers  2005), but does not expressly relate covariates to the occurrence of excess zeros (Minami et al. 2007).
                                                                                       
 %    	Mixture models such the zero inflated Poisson (ZIP) and zero inflated negative binomial (ZINB) (Zuur 2009, Cunningham and Lindenmayer 2005, Welsh et al. 2000): these models are useful for modelling counts of rare species when the number of zero observations is larger than expected. Zero inflated models are a process similar to the delta approach in which the presence or absence of the catch is modelled orthogonally to the size of the catch (Welsh et al 2000), however unlike the delta approach the count data can include zeros. These zeros could result from predator satiation, competition for hooks, or disinterest (called true zeros) as opposed to design errors, sampling errors, observer errors or zeros resulting from sampling outside the habitat range (called false zeros). The total probability of a zero count is then,
\subsubsection{Procedure for model selection}
\hlgreen{Laura or Joel to fix}
Initial model exploration began via fitting each model (species and population) with only one covariate and evaluating the explanatrory power of that covariate via AIC  (AIC values are listed in tables  listed in Table~\ref{tab:CPUE_AIC_1} throughTable~\ref{tab:CPUE_AIC_9} % cite AIC?Akaike, H. (1974), "A new look at the statistical model identification" (PDF), IEEE Transactions on Automatic Control 19 (6): 716-723, doi:10.1109/TAC.1974.1100705, MR 0423716.  get the table refs right 
Model fitting proceded by fitting additional coviariates in order of the proportion of deviance explained  on their own ( Table~\ref{tab:CPUE_AIC_1} through Table~\ref{tab:CPUE_AIC_9}).  Only those coviariates showing a change in AIC of 10 or greater were kept in the model.
%-- added covariate on MU ONLY (mean) until AIC showed decline $<$ 10 (arbitrary....) \\
If model diagnostics indicated a departure from the assumed error strucutre was present a covariate was added to $\sigma$ until the model mis-specification was resolved. Covariates were added based on the principle of a minimally efficient design and beginning with those coviariates for which the AIC~$\sigma$ table (Table~\ref{tab:CPUE_AIC_1} ) indicated had the most explanatory power.  
The addition of explicity modeling the $\sigma$ greatly improved the  model diagnostics in  most cases where the qqnorm plot indicated a misspecification in the tail of the assumed distribution.  The objectiv of the CPUE standardizations were to produce cpue values that accounted for changes in catch rate not due to changes in abundance over time, in this case the time step was an annual time step, therefore year effects were included in each model. 
All models also included either flag or observer program code, because flag and observer program are often highly correlated, one or the other as an explanatory categorical variable, based on the proportion of variance it explained and whether model diagnostics were impacted.
Interaction between year and observer program was also modeled in part to account for potential changes in individual program practices (e.g. reporting to species, banning shark fishing) within a particular program.  Additionally modeling year and program may account for localized trends in annual abundance, or sampling effort that are reflected in the program specific observer data. Other interaction effects were investigated early in the model selection process but discarded due to low explanatory power.%
% did I get this right?
%\hlgreen{but did not proceeded with an interaction for the remaining of the model selection}.
% huh? 
%as doing so would entail if the AIC
%score when interactions are allowed is at least 50 lower than with
%additive effects only.
                                                                                    
\subsubsection{Model diagnostics}                                                                                      
Model selection included examination of diagnostics including quantile residuals, normal quantile-quantile plots, plots of the  residuals residuals vs fitted factors, residuals  vs. explanatory factors as well as an overall summary.


\subsubsection{Calculation of year indices and confidence intervals}
Year effects could be extracted as is from the model output as there were no interactions between year and other variables, and year was not included as an explanatory variable for the $\sigma$ models. Confidence intervals were computed with the function \texttt{confint} in the R package \texttt{stats}\hlcoral{CITE}.
%Multiple methods of calculating the indices of abundance and confidence intervals exist depending on the model type (Shono H. 2008, Maunder and Punt 2004). In this study estimates were %calculated by predicting results based on the fitted model and a training data set that included each year effect and  the mean effect for each covariate (Zuur et al 2009). Confidence %intervals were calculated as ?1.96* SE, where SE is the standard error associated with the predicted year effect term.  Appendices hold the model diagnostics.

\clearpage
\subsection{Results}
%----------------------------------------------------------------------------------------

Nominal CPUE by species and region for longline sets are shown in Figures \ref{fig:nomCPUE} and \ref{fig:nomCPUErel} and for associated and unassociated purse seine sets in regions 3 and 4 in Figures \ref{fig:nomCPUEpsAss} and \ref{fig:nomCPUEpsUnass} respectively. Plots of nominal and standardised CPUE on a species by species basis are shown in Figures \ref{fig:blueStdCPUEnorth} to \ref{fig:thresherStdCPUE}.
 
\begin{description}
\item[Blue shark (\emph{Prionace glauca}), north Pacific] Both the standardized and nominal CPUEs of blue shark in the north Pacific show a declining trend starting in 1999 and 1998, respectively. Data points for 2011 and 2012 are unavailable due to low sample size.  
 
 \item[Blue shark (\emph{Prionace glauca}), south Pacific]  Both the standardized and nominal CPUEs for blue shark in the south Pacific show declines in the initial 1995-2003 and late 2010-2015 periods, with relatively stable CPUEs between 2004  and 2009.  
 
 \item[Mako shark (\emph{Isurus oxyrinchus} and \emph{Isurus paucus}) in the north Pacific] The standardized and nominal CPUEs share the same trajectories (Figure \ref{fig:makoStdCPUEnorth}), but on a slightly different scale for the first 6 years (1995-2001).  The largest difference in the nominal and standardized CPUE is in the final year, where the standardized CPUE declines sharply in contrast to the nominal, but years 2011 and 2012 were excluded from the standardization due to poor sample sizes (Figure \ref{makoStdCPUEvars2}).
 
\item[Mako shark (\emph{Isurus oxyrinchus} and \emph{Isurus paucus}) in the south Pacific] The standardized CPUEs show a more stable trend in relative abundance than the nominal CPUEs, although both have low points in 2002 and 2014. In addition, the standardized CPUE peaks in 2010, whereas the nominal is the highest in 1996.  
 
\item[Oceanic whitetip shark (\emph{Carcharhinus longimanus})] The standardized oceanic whitetip shark trend decreases steadily over 1995-2014.  The standardized trend shows a slightly steeper decline than the nominal, with the most noticeable departure from the nominal being the large decrease from 2013-2014 in the standardized CPUE.%   Diagnostics indicate no departure from the normality assumptions made in the mode.
 
 \item[Silky shark (\emph{Carcharhinus falciformis})] Standardized silky shark trends in the WCPO showed high inter-annual variability with an initial decline from 1995-2000 followed by a slight increase until 2010, followed by a steep decline. This mirrors the trends seen in the nominal CPUE, albeit with a lesser variability.
 
\item[ Thresher sharks (\emph{ Alopias superciliousus, vulpinus, \& pelagicus})]  Standardized CPUE values for thresher sharks were similar to the nominal CPUE except for additional variability in the nominals. They both rise for the first 6 years of the series (1995-2001) but diverge afterwards. For the years 2002-2014, the standardized CPUE is less variable showing a slightly decreasing CPUE from 2003-2011. The last three years of both the standardized and nominal CPUEs show a steep decline.  The CPUE from the thresher complex (bigeye, common and pelagic) is difficult to interpret as the second most commonly reported thresher species is the general ``thresher shark" category. 
 
\item[Hammerhead sharks (\emph{ Sphyrna mokarran, lewini, zygaena, \& Eusphyra blochii})]  Standardized CPUE for the hammerhead complex shows a large increase from the 3rd to the 6th year of the study period (1997-2001), with a relatively stable CPUE thereafter (2002-2013, regions 3 and 5 in the longline database). Similar to the thresher shark complex, the CPUE series representing the hammerhead complex are difficult to interpret because more than half of the observations in the study period (1995-2014) were made  to a generic ``hammerhead'' category.
 
 \item[Porbeagle shark (\emph{ Lamna nasus}) ] The standardized CPUE for porbeagle shark was close quite similar to the nominal CPUE, showing an increase in the first three years of the time series, followed by a decline from 1999 to 2003, and a monotonic increase thereafter.
 
 
%\item[ Whale Shark (\emph{ Rhincodon typus}) ] Whale shark interactions are generally reported only by the purse seine fishery. These observations are subject to considerable spatial and temporal heterogeneity and likely to have been affected by changes in observer coverage and reporting practices in recent years. The fate of whale sharks following interactions is also uncertain and information on key biological processes are limited. Given the current SPC data holdings only limited analysis for whale sharks in the WCPO is considered to be feasible.
 
\end{description}

%\addcenterfig[Blue shark CPUE indicators. Proportion of positive sets, observer data.]{fig:bshcp1}{FIG_xx_pcntpos_reg_BSH}
%\addcenterfig[Blue shark CPUE indicators. Nominal CPUE, sharks per 1000 hooks, observer data.]{fig:bshcp2}{FIG_xx_nomCPUE_reg_BSH}
%\addcenterfig[Blue shark CPUE indicators. Standardized blue shark CPUE based on the negative binomial model for observer data in the northern hemisphere.]{fig:bshcp3}{LLcpue_BSH_north_NB_step}
%\addcenterfig[Blue shark CPUE indicators. Standardized CPUE, zero inflated negative binomial Southern Hemisphere, observer data.]{fig:bshcp4}{ll_cpue_BSHzinb_nominal}

%----------------------------------------------------------------------------------------
          
%\addcenterfig[Mako shark CPUE indicators. Proportion of positive sets, observer data.]{fig:makcp1}{FIG_xx_pcntpos_reg_MAK}
%\addcenterfig[Mako shark CPUE indicators. Nominal CPUE, sharks per 1000 hooks, observer data.]{fig:makcp2}{FIG_xx_nomCPUE_reg_MAK}

%\addcenterfig[Mako shark CPUE indicators. Standardized CPUE, mako shark in the northern hemisphere.]{fig:makcp3}{LLcpue_MAKO_north_NB_cpue}
%\addcenterfig[Mako shark CPUE indicators. Standardized CPUE, mako shark in the southern hemisphere.]{fig:makcp4}{LLcpue_MAKO_south_NB_cpue}

%----------------------------------------------------------------------------------------

%\addcenterfig[Silky shark CPUE indicators. Proportion of positive sets, observer data.]{fig:falcp1}{FIG_xx_pcntpos_reg_FAL}
%\addcenterfig[Silky shark CPUE indicators. Nominal CPUE, sharks per 1000 hooks, observer data.]{fig:falcp2}{FIG_xx_nomCPUE_reg_FAL}

%\addcenterfig[Silky shark CPUE indicators. Standardized CPUE from longline observer data for silky sharks.]{fig:falcp3}{LLcpue_FAL_NB_cpue}

%----------------------------------------------------------------------------------------

%\addcenterfig[Oceanic whitetip shark CPUE indicators. Proportion of positive sets, observer data.]{fig:ocscp1}{FIG_xx_pcntpos_reg_OCS}
%\addcenterfig[Oceanic whitetip shark CPUE indicators. Nominal CPUE, sharks per 1000 hooks, observer data.]{fig:ocscp2}{FIG_xx_nomCPUE_reg_OCS}

%\addcenterfig[Oceanic whitetip shark CPUE indicators. Standardized CPUE based on negative  binomial models applied to observer data.]{fig:ocscp3}{LLcpue_OCS_NB_cpue}

%----------------------------------------------------------------------------------------
%\subsubsection{Thresher Shark}
          
%\addcenterfig[Thresher shark CPUE indicators. Proportion of positive sets, observer data.]{fig:thrcp1}{FIG_xx_pcntpos_reg_THR}
%\addcenterfig[Thresher shark CPUE indicators. Nominal CPUE, sharks per 1000 hooks, observer data.]{fig:thrcp2}{FIG_xx_nomCPUE_reg_THR}
%\addcenterfig[Thresher shark CPUE indicators.  Standardized CPUE of thresher shark based on longline observer data.]{fig:thrcp3}{LLcpue_THRESHER_NB_cpue}

%----------------------------------------------------------------------------------------          
      


      
\clearpage     
      
\begin{table}[!h]
\begin{center}
\caption{Summary of temperature ranges by species used as filters for cells to retain in the CPUE analysis. \label{meth:temprange}}
\begin{tabular}{l|c|c}
Species & Minimum T(\degree C) & Maximum T(\degree C)\\
\hline
\hline
Blue shark& 10 & 30\\
Hammerhead sharks& 13 & 30\\
Mako sharks& 11 & 30\\
Oceanic whitetip shark&18&30\\
Porbeagle shark&10&26\\
Silky shark&18&30\\
Thresher sharks&11&30\\
%Using range of 10 to 30 for BSH
%Using range of 18 to 30 for FAL
%Using range of 13 to 30 for HHD
%Using range of 11 to 30 for MAK
%Using range of 18 to 30 for OCS
%Using range of 10 to 26 for POR
%Using range of 11 to 30 for THR      
\end{tabular}
\end{center}
\end{table}
% include hbf by program_code
% dot map of catches over time
% include table of AIC change by variables on their own


\begin{table}[!h]
\caption{Description of variables used in CPUE standardization \label{tbl:glm-vars}}
\begin{center}
\begin{tabular}{l|l|p{7cm}|c}
Variable name & Symbol & Explanation & \% records present\\
\hline
\hline
Year & $\beta_Y$ & Required to estimate year effect & 100\\
Month & $\beta_M$ & Captures seasonal variability & 100\\
Observer program & $\beta_O$ & Country hosting the observer program & 100\\
Vessel flag & $\beta_F$ & Note: correlated with observer program & 100\\
Hooks-beween-floats& $\beta_{HBF}$ & Indicator of catchability for surface-dwelling species\\
Shark bait &&\\
%Number of shark lines&&\\
%Lighsticks &&\\
Shark target&&Sharks explicitly defined as targets?\\
SST & $SST$ & 100\\
Day category && Day or night, before or after sunrise/sunset?&100\\

\end{tabular}
\end{center}
\end{table}


%in appendix: summarize proportion of data removed for each species (temperature, observer, quantile), e.g. North Mako: removing HW removes a lot of data!! check for catches...

\subsection{Model diagnostics}
The quality of model fit was assessed by inspecting the residuals as well as simulating data from the fitted GLM model and comparing the simulated to the observed. 
Quantile residuals \citep{Dunn1996_a} were used instead of the traditional deviance or Pearson's residuals. Quantile residuals overcome many of the problems encountered for count-based GLMs and greatly facilitate model diagnostics. More specifically we aimed to improve fit to both the zeros and high catch events. Diagnostics were very satisfactory for all stocks except that of southern blue shark for which high catch events were still not well modelled, presumably due to unaccounting targeting by some fleets. Key diagnostics are included in \hlcoral{Figures XXX to XXX}.


\subsection{Conclusions}
\hlgreen{re-read: high priority}
The signals from the nominal  CPUE data can be heavily influenced by factors other than abundance and therefore a procedure to standardize CPUE data for changes in factors  that do not reflect changes in abundance is usually recommended. 

This is because accounting for the large amount of zeroes in shark CPUE catch data often comes at the expense of modelling large catch events, 
 
Analysing and interpreting CPUE trends for highly mobile species on a population level based on a small subset of the actual encounters with the fishery can be difficult, a number of potential biases can complicate this analysis   These biases arise either from changes in the fisheries themselves (e.g. operational or gear changes) or from changes in observer coverage of these fisheries (e.g. observer data not provided for some years) or from the species interactions with natural occurring forcing factors (e.g. el nino).  Many of these issues and their potential effect on sharks are documented in the previous report (Clarke et al 2011).  %last report.
Briefly the issues can be broken down into four main areas:
\begin{description}
%\begin{itemize}
\item[Targeting:] Evidence of shark targeting and mixed shark/tuna or shark/billfish  exists in many regions including region 1, 3 and 4. 
\item[Changes in regulations]: Changes in regulations with respect to banning of shark finning by US vessels and in US waters in 2000; the closure of the shallow set component of the Hawaii longline fishery for three years beginning 2001; banning of  finning by Australia in 2000, and the banning of wire leaders by Australia in 2005. The banning of shark fishing (except for makos) by French Polynesia in 2006
\item[Changes in data availability:] Since the implementation of the WCPFC ROP on 1 January 2010 100\% observer coverage has been required on the purse seine fleet. In practice this reduced the number of observers on longline vessels as they switched to purse seine vessels. This created discontinuities in the observer coverage for some species.
%\end{itemize}
\end{description}
Despite these obstacles, analysis of the catch rate (both nominal and standardizes) of the key species revealed that ;

Blue shark CPUE is declining in the north and south pacific based on the nominal and standardized CPUE.

Oceanic white tip continue to decline throughout the tropical waters of the WCPO

Silky shark cpue exhibits high fluctuations throughout the study period.


The thresher shark complex appears to be in decline though the last years data point is based on relatively few data.

Mako shark in the south pacific may have been declining for the last five years however the last data point is based on relativly few data points. 
% intresting that HPBCAT2 changes nearly linearly across the study time  

Mako in the north Pacific is similarily plauged by data deficiencies missing dta for years 1996, 2003, 2011, and 2012. Despite this the trend looked relatively stable between 2000 and 2010, however no infrence is possible for the lst 4 years.

Porbeagle shark CPUE experienced a large decrease early on in the study period followed by a fluctuating but increasing CPUE trend.

In addition: Error distributions for by-catch species have been discussed at length in previous publications as these data are notoriously hard to model properly due to the high proportion of zeroes \citep{...}. 

\clearpage
\begin{landscape}
\thispagestyle{empty}
 \begin{table}[!h]
\caption{Summary of number of records removed by filter type for each species before GLM analyses. HW/AS and PG refer to the Hawaai/American Samoa and the Papua New Guinea observer programs. OB sampling refers to records removed from observer programs with few records. See summary in section \ref{cpuemeth:datafilter} \label{tbl:glmdata}}
\begin{center}
\begin{tabular}{l|c|c|c|c|c|c|c|c}
Species & Hemisphere & SST range & max quantile & HW/AS & PG & OB sampling & \# rows left\\
\hline
\hline
Blue shark, south&41276&1234&309&3449&571&21&19660\\ 
Blue shark, north&25244&0&805&36818&0&35&3618\\ 
Hammerhead sharks&0&4999&12&41072&571&21&19845\\ 
Mako sharks, south&41276&1419&130&3359&571&21&19744\\ 
Mako sharks, north&25244&0&97&37536&0&35&3608\\ 
Oceanic whitetip shark&0&10266&171&38532&570&21&16960\\ 
Porbeagle shark&41276&18038&78&0&0&122&7006\\ 
Silky shark&0&10266&127&38563&563&21&16980\\ 
Thresher sharks&0&1419&290&40860&571&21&23359\\

 
\end{tabular}
\end{center}
\end{table}
%%%%%%%%%%%%%%%%%%%%%%%%%%%%%%%%%%%%%%%%%%%%%%%
\begin{table}[!h]
\caption{Summary of model structures retained for CPUE standardization of each species}
\begin{center}
\begin{tabular}{l|c|c|c}
Species & Model $\mu$& model $\sigma$ & \% deviance\\
\hline
\hline
Blue shark, northern stock  & year + program + HPBCAT2 + month + sharkbait & program + HPBCAT2 + month&\\
Blue shark, southern stock & year + flag + HPBCAT2 + month + sharkbait & \\ 
Mako, southern stock & year + program + HPBCAT2 + month + sharkbait & program + month & ---\\
Mako, northern stock & year + program + month + sharkbait & HPBCAT2 \\
Oceanic white tip & year + program + HBPCAT2 + month + sharkbait & program \\
Thresher sharks& year + program + HPBCAT2 + month & program + HBPCAT2 \\
Hammerheads& year + program + HPBCAT2 + month + sharkbait & sharkbait \\
Oceanic whitetip shark& year + program +HBPCAT2 + month + sharkbait & program\\
Silky shark & program + year + HPBCAT2 + month + sharkbait & program + sharkbait + month\\
Porbeagle & year + flag + HPBCAT2 + month & flag + month\\ 

\end{tabular}
\end{center}
\end{table}
\end{landscape}



%\end{document}
%\documentclass{SCreport}

%\begin{document}
%%%%%%%%% Defining latex macros to print tables %%%%%%%%%%


\newcommand{\blueaic}{
% latex table generated in R 3.2.0 by xtable 1.7-4 package
% Sat Jul 04 14:52:03 2015
\begin{table}[ht]
\centering
\caption{AIC improvement over null model for blue from a single explanatory variable} 
\label{blue:aic1}
\begin{tabular}{lrr}
  \hline
Variable & AIC.diff & AIC.diff.sigma \\ 
  \hline
program\_code & 24382.47 & 16607.68 \\ 
  flag\_id & 23711.29 & 17163.94 \\ 
  HPBCAT2 & 20183.28 & 5765.59 \\ 
  HPBCAT & 20178.74 & 5576.42 \\ 
  yy & 14357.75 & 8612.54 \\ 
  mm & 3725.06 & 2511.55 \\ 
  sharkbait & 994.97 & 454.07 \\ 
   \hline
\end{tabular}
\end{table}

}


\newcommand{\makonorthaic}{
% latex table generated in R 3.2.0 by xtable 1.7-4 package
% Sat Jul 04 14:52:03 2015
\begin{table}[ht]
\centering
\caption{AIC improvement over null model for mako.north from a single explanatory variable} 
\label{mako.north:aic1}
\begin{tabular}{lrr}
  \hline
Variable & AIC.diff & AIC.diff.sigma \\ 
  \hline
HPBCAT2 & 6433.14 & 642.33 \\ 
  HPBCAT & 6417.24 & 640.61 \\ 
  mm & 2845.80 & 216.50 \\ 
  yy & 2216.05 & 401.91 \\ 
  flag\_id & 115.74 & 61.51 \\ 
  program\_code & 114.17 & 44.28 \\ 
  sharkbait & 15.46 & 3.29 \\ 
   \hline
\end{tabular}
\end{table}

}


\newcommand{\makosouthaic}{
% latex table generated in R 3.2.0 by xtable 1.7-4 package
% Sat Jul 04 14:52:03 2015
\begin{table}[ht]
\centering
\caption{AIC improvement over null model for mako.south from a single explanatory variable} 
\label{mako.south:aic1}
\begin{tabular}{lrl}
  \hline
Variable & AIC.diff & AIC.diff.sigma \\ 
  \hline
program\_code & 1693.97 & 251.6 \\ 
  flag\_id & 1438.39 & --- \\ 
  HPBCAT2 & 1202.23 & 79.45 \\ 
  HPBCAT & 1127.42 & 34.67 \\ 
  yy & 875.66 & 389.5 \\ 
  mm & 733.11 & 178.28 \\ 
  sharkbait & 8.37 & -1.89 \\ 
   \hline
\end{tabular}
\end{table}

}


\newcommand{\ocsaic}{
% latex table generated in R 3.2.0 by xtable 1.7-4 package
% Sat Jul 04 14:52:03 2015
\begin{table}[ht]
\centering
\caption{AIC improvement over null model for ocs from a single explanatory variable} 
\label{ocs:aic1}
\begin{tabular}{lrr}
  \hline
Variable & AIC.diff & AIC.diff.sigma \\ 
  \hline
yy & 3726.21 & 958.91 \\ 
  program\_code & 2781.62 & 1369.09 \\ 
  flag\_id & 1499.89 & 772.13 \\ 
  sharkbait & 887.50 & 103.91 \\ 
  HPBCAT2 & 308.22 & 1656.57 \\ 
  HPBCAT & 256.09 & 1658.57 \\ 
  mm & 143.92 & 185.56 \\ 
   \hline
\end{tabular}
\end{table}

}


\newcommand{\silkyaic}{
% latex table generated in R 3.2.0 by xtable 1.7-4 package
% Sat Jul 04 14:52:03 2015
\begin{table}[ht]
\centering
\caption{AIC improvement over null model for silky from a single explanatory variable} 
\label{silky:aic1}
\begin{tabular}{lrr}
  \hline
Variable & AIC.diff & AIC.diff.sigma \\ 
  \hline
program\_code & 10913.31 & 6886.75 \\ 
  flag\_id & 10144.24 & 5446.47 \\ 
  sharkbait & 5382.46 & 1713.24 \\ 
  yy & 4986.96 & 3273.31 \\ 
  HPBCAT2 & 1639.75 & 4527.65 \\ 
  HPBCAT & 1628.44 & 4513.75 \\ 
  mm & 957.50 & 358.96 \\ 
   \hline
\end{tabular}
\end{table}

}



\subsection{Introduction}

%This paper follows from the previous indicator based analysis presented to the Western and Central Pacific Fisheries Commission (WCPFC) Scientific Committee (SC7, Clarke et al. 2011), stock assements (Rice et al. 2014, Rice et al.2013, Rice et al.2012) \hllb{(cite the standardization papers, cite ISC work?)}. The developments presented here include additional analyses of the Secretariat of the Pacific (SPC) data holdings for silky caught in longline and purse seine fisheries in the Western and Central Pacific Ocean (WCPO), though we note that some previous data (Japan) was not available for this effort. Standardized catch per unit of effort (CPUE) series are developed for the main shark species.  
%\emph{Note: The current analysis does not construct inputs to use for stock assessments or catch estimates. Our goal is to highlight general trends in population abundance over time, to be interpreted together with other indicators as outlined above. We recommend that catch rates standardization for stock assessments or catch estimates be conducted independently.}


% intro      
Catch-per-unit-effort (CPUE) data are commonly used as indices of abundance for marine species. However, multiple factors---fishing technique, season, bait type, etc---can alter the relationship between CPUE and abundance, especially in complex fisheries systems comprising of several fleets and spanning large spatial and temporal scales. Nominal catch rates must therefore be standardized to account for changes in these factors over time. This is typically done using General Linear Models (GLM), which allow us to model the relationship between CPUE and a set of explanatory variables. The dataset used in this analysis provided many candidate variables, but, given the diversity of observer programs represented, few had enough coverage to be retained in the final models. The available variables are described in Table \ref{tbl:glm-vars} (see also Table 2 in \citet{Francis2014_b} for an overview of the use of variables in shark CPUE standardisations).

% issue with sharks cpue standrz
CPUE data for sharks often have a large proportion of observations (sets) with zero shark catch, while while some sets have large catch. These instances of high catch can occur when areas of high shark densities are accidentally encountered or when fishers target sharks. The co-existence of both high proportions of zeros and high catch results in over-dispersed data, typical of bycatch species. These features are challenging to account for from a statistical point of view, and have been reviewed at length in the literature on bycatch analyses \citep{Bigelow2002_a,Campbell2004_a,Ward2005_a,Minami2007_a}. 
%(Bigelow et al. 2002; Campbell 2004, Ward and Myers 2005; Minami et al. 2007).

Zero-inflated approaches have been advocated as best suited to model over-dispersion in both zeros and positive counts (Brodziak et al., 2013). A drawback of the zero inflated approach is that it is data intensive and models fail to converge. In addition, in practice it tends to be more successful at modelling the extra zeroes than the large-catch events, since the dispersion parameter which controls the length of the tail is assumed to be constant over all factors. This is especially a problem when the mean of the distribution is close to zero or one, as in those instances the probability of getting large events if mostly controlled by the dispersion parameter $\sigma$, unlike when the mean is larger and the tail is not as pronounced. However, whenever conditions are good for sharks and/or targeting takes place, larger catch events can happen and not modelling those properly implies that we are missing important drivers of catch. Typically, this can be diagnosed as a departure from the one-to-one line on the right-hand side of quantile-quantile plots. Here we achieved significant improvements in model diagnostics by keeping a negative binomial distribution but allowing both the mean $\mu$ and $\sigma$ parameters to be modelled against covariates \citep{Rigby2005_a}. This approach is much less computationally intensive than zero-inflated models, and yielded excellent diagnostics with these data.


 \subsection{Methods}
Standardised CPUE series for the longline bycatch fisheries were developed using generalised linear models using longline catch and effort data. The number of hooks in a longline set was used as a measure of effort.
See also \citet{Clarke2011_a, Clarke2011_b},  \citet{Walsh2011_a}, \citet{Rice2012_a}   for past work on shark CPUE standardisation in the Western and Central Pacific Ocean (WCPO).
%  @Article{,
 %   title = {Generalized additive models for location, scale and shape,(with discussion)},
 %   journal = {Applied Statistics},
 %   volume = {54},
 %   part = {3},
 %   pages = {507-554},
 %   year = {2005},
 %   author = {R. A. Rigby and D. M. Stasinopoulos},
 % }
 \label{cpuemeth:datafilter}
 \subsubsection{Stock definition for the purpose of the analysis}
   
% talk about stocks and year span                                                                                        
%Silky and oceanic white tip sharks are observed mainly in the equatorial waters in the purse seine fishery (Figure \ref{...}), and from about -25\degree~S to 25\degree~N in the longline fishery (Figure 1). 
Silky and oceanic whitetip sharks have each been assessed \citep{Rice2012_a, Rice2013_a} as single stocks in the WCPO, and are presented in this analysis as a single stock.  Thresher, mako and blue sharks occur more frequently in cold, temperate waters, and generally believed to be separated into northern and southern stocks. For instance, blue sharks in the North Pacific have been subject to multiple stock assessments as a stock unit. These temperate species are thus analysed as individual stocks. In the Pacific porbeagle sharks are only found south of 20\degree{}S and were analysed as a single stock. Hammerhead sharks for the purpose of this analysis (and throughout the report) are considered to be a single stock.

\subsubsection{Data Preparation}
To further define the expected geographic range, we defined a coarse climate `envelope' based on sea surface temperature. This aided in distinguishing between zero catch in areas where the species does not occur from zero catch in areas where the species occurs but was not caught. Temperature data were downloaded from the GODAS database (\url{www.cdc.noaa.gov/data/gridded/data.godas.html}) and matched to the observer data on a set by set basis. The temperature range of a species was defined as the minimum and maximum of the monthly mean sea surface temperatures (SST) of cells with positive catch for that species (see Table \ref{meth:temprange}). The temperature predicted by GODAS at the 5 meters depth at set time/day was used for the SST value. Only cells for which all mean monthly temperatures fell within this range were retained. %\hllb{REWORK. Note that this is a minimal filter and it was only used to exclude the most improbable cells from which a species could be seen.}
 %\hl{Looked at range of SST where positive catches occured, selected cells where median falls within this range.}
 
%                                                                        section \ref{app:datacleaning}     
          
Data were cleaned following the filters outlined in Table \ref{tbl:datfiltering}. Records from the US observer programs (Hawaii and American Samoa) were excluded from the analysis as they were only available up to 2011. Records from any observer programs for which we had less than 100 sets were removed. Extreme catch events greater than the $97.5^{th}$ quantile of the nominal CPUE were also removed. Finally, year effects were only estimated if there were at least 50 sets observed in that year (by species and population).  Records from the Papua New Guinea observer program were removed as vessels in the fleet frequently target sharks. Additional filters are listed by species in Table \ref{tbl:glmdata}.
                                                                                      
%Previous work \citep{Rice2012_a, Rice2013_a, Bromhead2012_a} has examined the extent of shark targeting in the WCPO tuna longline %fleet, and found that a number of factors (i.e. sharklines, wire trace can lead to a higher catch rates). Shark target sets comprise a much smaller proportion of the overall dataset (6.5\% of the sets), however, these sets represent significant shark catch for some species (e.g. 82\% of the total silky shark catch). Sets marked by observers as targeting sharks were therefore removed from the analysis as catch rates therein behave differently than when sharks are caught as by-catch. 
%Shark targeting sets were deemed to be sets where the observer had marked that the set was intentionally targeting sharks of any species, whether shark bait was used, or whether shark lines were used. We also removed the data from the Papua New Guinea observer program because of frequent shark targeting which may not always appear in observer records. 
                                                                                       
% \subsection{Additional categorical variables}    % jr edits don't think we need that
% 1. Day category: sunset and sunrise were calculated\\ 
% 2. HPBCAT: based on data exploration two types explored: either shallow or deep (with shallow $<=$ 10), or HPBCAT2 shallow, intermediate and deep, with intermediate between 10 and 15 ... in general hpbcat2 had slighlty better aic %(in tables at the end)
     %Latitude and longitude were truncated to the nearest 1\degree; this location information was used to calculate the set specific association with a 5\degree square (referred to hereinafter as cell). Date of set was used to calculate the year, month, quarter and trimester of the set. Set time was used to calculate the time category of the day in sixths starting at midnight. A non-target data set was created as a result of filtering data sets according to the above rules as well as filtering sets where sharks were the intentional target. This was done under the premise that the factors leading to non-zero catch rates when targeting sharks would be different than factors that lead to non-zero catch rates when not targeting sharks. 
                                                                                       
                                                                                       
   \subsection{Overview of GLM Analyses}
   \subsubsection{Notes on error distributions:} 
  
   The  filtered datasets  were standardised using generalized linear models \citep{McCullagh1989_a} with the software package R (\url{www.r-project.org}, \citealt{RCT2013_a}) and the package \texttt{gamlss} \citep{Rigby2005_a}. Initially, multiple assumed error structures were tested including the delta lognormal approach (DLN)   \cite{Lo1992_a, Dick2006_a, Stefansson1996_a}, zero-inflated poisson and negative binomial models, the tweedie distribution \citep{Shono2008_a} and negative binomial models with mu and  $\sigma$ modelled. Due to its superior performance both in run time and model diagnostics, we retained the latter and only present those results here.
%  in JR_bib_edts.bak Hiroshi Shono 2008. Application of the Tweedie distribution to zero-catch data in CPUE analysis. Fisheries Research 93 (2008) 154-162

%The negative binomial (Lawless, 1987) is typically more robust to issues of over-dispersion (over-dispersion can arise due to excess zeros, clustering of observations, or from correlations between observations) was also used. This model has been advocated as a model that is more robust to over-dispersion than the Poisson distribution (McCullagh and Nelder 1991), and is appropriate for count data (Ward and Myers  2005), but does not expressly relate covariates to the occurrence of excess zeros (Minami et al. 2007).
                                                                                       
 %    	Mixture models such the zero inflated Poisson (ZIP) and zero inflated negative binomial (ZINB) (Zuur 2009, Cunningham and Lindenmayer 2005, Welsh et al. 2000): these models are useful for modelling counts of rare species when the number of zero observations is larger than expected. Zero inflated models are a process similar to the delta approach in which the presence or absence of the catch is modelled orthogonally to the size of the catch (Welsh et al 2000), however unlike the delta approach the count data can include zeros. These zeros could result from predator satiation, competition for hooks, or disinterest (called true zeros) as opposed to design errors, sampling errors, observer errors or zeros resulting from sampling outside the habitat range (called false zeros). The total probability of a zero count is then,
\subsubsection{Procedure for model selection}
% cite AIC?Akaike, H. (1974), "A new look at the statistical model identification" (PDF), IEEE Transactions on Automatic Control 19 (6): 716-723, doi:10.1109/TAC.1974.1100705, MR 0423716.  get the table refs right 
Initial model exploration began via fitting each model (species and population) with only one covariate and evaluating the explanatory power of that covariate \via AIC (AIC values are listed in tables  listed in Tables~\ref{BSH.north:aic1} to \ref{THR:aic1}. Model fitting proceeded by fitting additional covariates in order of the proportion of deviance explained  on their own.  Only those covariates showing a change in AIC of 10 or greater were kept in the model.
%-- added covariate on MU ONLY (mean) until AIC showed decline $<$ 10 (arbitrary....) \\

If model diagnostics indicated a departure from the assumed error structure, covariates were added to $\sigma$ until the model mis-specification was resolved. Covariates were added based on the principle of a minimally efficient design and beginning with those covariates for which the AIC~$\sigma$ table (Tables \ref{BSH.north:aic1} to \ref{THR:aic1}) indicated had the most explanatory power.  

The addition of explicitly modelling the $\sigma$ greatly improved the  model diagnostics in  most cases where the Q-Q norm plot indicated a misspecification in the tail of the assumed distribution.  The objective of the CPUE standardisations were to produce CPUE values that accounted for changes in catch rate not due to changes in abundance over time, in this case the time step was an annual time step, therefore year effects were included in each model. 
All models also included either flag or observer program code, because flag and observer program are often highly correlated, one or the other as an explanatory categorical variable, based on the proportion of variance it explained and whether model diagnostics were impacted.
Interaction between year and observer program was also modeled in part to account for potential changes in individual program practices (e.g. reporting to species, banning shark fishing) within a particular program.  Additionally modelling year and program may account for localised trends in annual abundance, or sampling effort that are reflected in the program specific observer data. Other interaction effects were investigated early in the model selection process but discarded due to low explanatory power.%
% did I get this right?
%\hlgreen{but did not proceeded with an interaction for the remaining of the model selection}.
% huh? 
%as doing so would entail if the AIC
%score when interactions are allowed is at least 50 lower than with
%additive effects only.
                                                                                    
\subsubsection{Model diagnostics}                                                                                      
Model selection included examination of diagnostics including quantile residuals \citep{Dunn1996_a}, normal quantile-quantile plots, plots of the residuals residuals vs fitted factors, residuals vs. explanatory factors as well as an overall summary.
The quality of model fit was assessed by inspecting the residuals as well as simulating data from the fitted GLM model and comparing the simulated to the observed. Quantile residuals  were used instead of the traditional deviance or Pearson’s residuals. Quantile residuals overcome many of the problems encountered for count-based GLMs and greatly facilitate model diagnostics. Specifically we aimed to improve fit to both the zeros and high catch events. Diagnostics were satisfactory for all stocks except that of blue shark in the south Pacific for which high catch events were still not well modelled, presumably due to unaccounted targeting by some fleets. Key diagnostics are included in Figures \ref{fig:diagno1} to \ref{fig:diagno18}.


\subsubsection{Calculation of year indices and confidence intervals}
Year effects could be extracted directly from the model output as there were no interactions between year and other variables, and year was not included as an explanatory variable for the $\sigma$ models. Confidence intervals were computed with the function \texttt{confint} in the R package \texttt{stats}.
%Multiple methods of calculating the indices of abundance and confidence intervals exist depending on the model type (Shono H. 2008, Maunder and Punt 2004). In this study estimates were %calculated by predicting results based on the fitted model and a training data set that included each year effect and  the mean effect for each covariate (Zuur et al 2009). Confidence %intervals were calculated as ?1.96* SE, where SE is the standard error associated with the predicted year effect term.  Appendices hold the model diagnostics.
%%%%%
\clearpage
\subsection{Results}
%----------------------------------------------------------------------------------------

Nominal CPUE by species and region for longline sets are shown in Figures \ref{fig:nomCPUE} and \ref{fig:nomCPUErel} and for associated and unassociated purse seine sets in regions 3 and 4 in Figures \ref{fig:nomCPUEpsAss} and \ref{fig:nomCPUEpsUnass} respectively. Plots of nominal and standardised CPUE on a species by species basis are shown in Figures \ref{fig:blueStdCPUEnorth} to \ref{fig:thresherStdCPUE}.
 
\begin{description}
\item[Blue shark (\emph{Prionace glauca}), north Pacific] Both the standardised and nominal CPUEs of blue shark in the north Pacific show a declining trend starting in 1999 and 1998, respectively. Data points for 2011 and 2012 are unavailable due to low sample size.  
 
 \item[Blue shark (\emph{Prionace glauca}), south Pacific]  
 Both the standardised and nominal CPUEs for blue shark in the south Pacific show declines in the initial 1995-2003 and late 2010-2015 periods, with relatively stable CPUEs between 2004  and 2009.  
 
 \item[Mako shark complex (\emph{Isurus oxyrinchus} and \emph{I. paucus}) in the north Pacific] 
 The standardised and nominal CPUEs share the same trajectories (Figure \ref{fig:makoStdCPUEnorth}), but on a slightly different scale for the first 6 years (1995-2001).  The largest difference in the nominal and standardised CPUE is in the final year, where the standardised CPUE declines sharply in contrast to the nominal, but years 2011 and 2012 were excluded from the standardisation due to poor sample sizes (Figure \ref{fig:diagno11}).
 
\item[Mako shark complex (\emph{Isurus oxyrinchus} and \emph{I. paucus}) in the south Pacific] 
The standardised CPUEs show a more stable trend in relative abundance than the nominal CPUEs, although both have low points in 2002 and 2014. In addition, the standardised CPUE peaks in 2010, whereas the nominal is the highest in 1996.
 
\item[Oceanic whitetip shark (\emph{Carcharhinus longimanus})] 
The standardised oceanic whitetip shark trend decreases steadily over 1995-2014.  The standardised trend shows a slightly steeper decline than the nominal, with the most noticeable departure from the nominal being the large decrease from 2013-2014 in the standardized CPUE.%   Diagnostics indicate no departure from the normality assumptions made in the mode.
 
 \item[Silky shark (\emph{Carcharhinus falciformis})] 
 Standardised silky shark trends in the WCPO showed high inter-annual variability with an initial decline from 1995-2000 followed by a slight increase until 2010, followed by a steep decline. This mirrors the trends seen in the nominal CPUE, albeit with a lesser variability.
 
\item[ Thresher shark complex (\emph{Alopias superciliousus, A. vulpinus, and A. pelagicus})]  
Standardised CPUE values for thresher sharks were similar to the nominal CPUE except for additional variability in the nominals. They both rise for the first 6 years of the series (1995-2001) but diverge thereafter. For 2002-2014, the standardised CPUE is less variable, decreasing slightly from 2003-2011. The last three years of both the standardised and nominal CPUEs show a steep decline.  The CPUE from the thresher complex is difficult to interpret as the second most commonly reported thresher species is the general ``thresher shark" category. 
 
\item[Hammerhead shark complex (\emph{Sphyrna mokarran, S. lewini, S. zygaena, and Eusphyra blochii})]  
Standardised CPUE for the hammerhead complex shows a large increase from the 3rd to the 6th year of the study period (1997-2001), with a relatively stable CPUE thereafter (2002-2013, regions 3 and 5 in the longline database). Similar to the thresher shark complex, the CPUE series representing the hammerhead complex are difficult to interpret because more than half of the observations in the study period (1995-2014) were recorded into the generic ``hammerhead'' category.
 
 \item[Porbeagle shark (\emph{ Lamna nasus}) ] 
 The standardised CPUE for porbeagle shark was close quite similar to the nominal CPUE, showing an increase in the first three years of the time series, followed by a decline from 1999 to 2003, and a monotonic increase thereafter.
 
 
%\item[ Whale Shark (\emph{ Rhincodon typus}) ] Whale shark interactions are generally reported only by the purse seine fishery. These observations are subject to considerable spatial and temporal heterogeneity and likely to have been affected by changes in observer coverage and reporting practices in recent years. The fate of whale sharks following interactions is also uncertain and information on key biological processes are limited. Given the current SPC data holdings only limited analysis for whale sharks in the WCPO is considered to be feasible.
 
\end{description}

%\addcenterfig[Blue shark CPUE indicators. Proportion of positive sets, observer data.]{fig:bshcp1}{FIG_xx_pcntpos_reg_BSH}
%\addcenterfig[Blue shark CPUE indicators. Nominal CPUE, sharks per 1000 hooks, observer data.]{fig:bshcp2}{FIG_xx_nomCPUE_reg_BSH}
%\addcenterfig[Blue shark CPUE indicators. Standardized blue shark CPUE based on the negative binomial model for observer data in the northern hemisphere.]{fig:bshcp3}{LLcpue_BSH_north_NB_step}
%\addcenterfig[Blue shark CPUE indicators. Standardized CPUE, zero inflated negative binomial Southern Hemisphere, observer data.]{fig:bshcp4}{ll_cpue_BSHzinb_nominal}

%----------------------------------------------------------------------------------------
          
%\addcenterfig[Mako shark CPUE indicators. Proportion of positive sets, observer data.]{fig:makcp1}{FIG_xx_pcntpos_reg_MAK}
%\addcenterfig[Mako shark CPUE indicators. Nominal CPUE, sharks per 1000 hooks, observer data.]{fig:makcp2}{FIG_xx_nomCPUE_reg_MAK}

%\addcenterfig[Mako shark CPUE indicators. Standardized CPUE, mako shark in the northern hemisphere.]{fig:makcp3}{LLcpue_MAKO_north_NB_cpue}
%\addcenterfig[Mako shark CPUE indicators. Standardized CPUE, mako shark in the southern hemisphere.]{fig:makcp4}{LLcpue_MAKO_south_NB_cpue}

%----------------------------------------------------------------------------------------

%\addcenterfig[Silky shark CPUE indicators. Proportion of positive sets, observer data.]{fig:falcp1}{FIG_xx_pcntpos_reg_FAL}
%\addcenterfig[Silky shark CPUE indicators. Nominal CPUE, sharks per 1000 hooks, observer data.]{fig:falcp2}{FIG_xx_nomCPUE_reg_FAL}

%\addcenterfig[Silky shark CPUE indicators. Standardized CPUE from longline observer data for silky sharks.]{fig:falcp3}{LLcpue_FAL_NB_cpue}

%----------------------------------------------------------------------------------------

%\addcenterfig[Oceanic whitetip shark CPUE indicators. Proportion of positive sets, observer data.]{fig:ocscp1}{FIG_xx_pcntpos_reg_OCS}
%\addcenterfig[Oceanic whitetip shark CPUE indicators. Nominal CPUE, sharks per 1000 hooks, observer data.]{fig:ocscp2}{FIG_xx_nomCPUE_reg_OCS}

%\addcenterfig[Oceanic whitetip shark CPUE indicators. Standardized CPUE based on negative  binomial models applied to observer data.]{fig:ocscp3}{LLcpue_OCS_NB_cpue}

%----------------------------------------------------------------------------------------
%\subsubsection{Thresher Shark}
          
%\addcenterfig[Thresher shark CPUE indicators. Proportion of positive sets, observer data.]{fig:thrcp1}{FIG_xx_pcntpos_reg_THR}
%\addcenterfig[Thresher shark CPUE indicators. Nominal CPUE, sharks per 1000 hooks, observer data.]{fig:thrcp2}{FIG_xx_nomCPUE_reg_THR}
%\addcenterfig[Thresher shark CPUE indicators.  Standardized CPUE of thresher shark based on longline observer data.]{fig:thrcp3}{LLcpue_THRESHER_NB_cpue}

%----------------------------------------------------------------------------------------          
      


      

%in appendix: summarize proportion of data removed for each species (temperature, observer, quantile), e.g. North Mako: removing HW removes a lot of data!! check for catches...


\subsection{Conclusions}

The signals from the nominal  CPUE data can be heavily influenced by factors other than abundance, and therefore a procedure to standardise CPUE data for changes in factors  that do not reflect changes in abundance is recommended. 

%This is because accounting for the large amount of zeroes in shark CPUE catch data often comes at the expense of modelling large catch events, 
 
Analysing and interpreting CPUE trends for highly mobile species on a population level based on a small subset of the actual encounters with the fishery can be difficult, a number of potential biases can complicate this analysis   These biases arise either from changes in the fisheries themselves (e.g. operational or gear changes) or from changes in observer coverage of these fisheries (e.g. observer data not provided for some years) or from the species interactions with natural occurring forcing factors (e.g. El Ni\~no). Many of these issues and their potential effect on sharks are documented in the previous report \citep{Clarke2011_a}.  %last report.
Briefly the issues can be broken down into three main areas:
\begin{description}
\item[Targeting:] There is evidence of shark targeting and mixed shark/tuna or shark/billfish fisheries in many regions including regions 1, 3 and 4. 
\item[Changes in regulations:] Changes in regulations that will impact CPUE indices include: banning of shark finning by US vessels and in US waters in 2000; the closure of the shallow set component of the Hawaii longline fishery for three years beginning 2001; banning of  finning by Australia in 2000; banning of wire leaders by Australia in 2005; and the banning of shark fishing (except for makos) by French Polynesia in 2006.
\item[Changes in data availability:] Since the implementation of the WCPFC ROP on 1 January 2010 100\% observer coverage has been required on the purse seine fleet. In practice this reduced the number of observers on longline vessels as they were transferred to the purse seine fleet. This created discontinuities in the observer coverage for some species.
\end{description}

Despite these obstacles, analysis of the catch rate (both nominal and standardised) of the key species revealed that:
\begin{itemize}

\item Blue shark CPUE is declining in the north and south pacific based on the nominal and standardised CPUE.

\item Oceanic whitetip continue to decline throughout the tropical waters of the WCPO.

\item Silky shark CPUE exhibit high fluctuations throughout the study period.

\item The thresher shark complex appears to be declining though the last data point is based on relatively few data and may exaggerate the trend in the last year.

\item Mako shark in the south Pacific may have been declining for the last five years, however, 2014 is based on relatively few data and may therefore be unreliable. Mako in the north Pacific is similarly plagued by data deficiencies with not enough data to estimate year effects in 1996, 2003, 2011, and 2012. Despite this, the trend looked relatively stable between 2000 and 2010, however, no inference is possible for the last 4 years.% intresting that HPBCAT2 changes nearly linearly across the study time  

\item Porbeagle shark CPUE experienced a large decrease until 2003 and has increased since then.

\end{itemize}

%In addition: Error distributions for by-catch species have been discussed at length in previous publications as these data are notoriously hard to model properly due to the high proportion of zeroes .%\citep{...}. 



%\end{document}
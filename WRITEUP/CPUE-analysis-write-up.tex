\documentclass{SCreport}

\begin{document}
%%%%%%%%% Defining latex macros to print tables %%%%%%%%%%


\newcommand{\blueaic}{
% latex table generated in R 3.2.0 by xtable 1.7-4 package
% Sat Jul 04 14:52:03 2015
\begin{table}[ht]
\centering
\caption{AIC improvement over null model for blue from a single explanatory variable} 
\label{blue:aic1}
\begin{tabular}{lrr}
  \hline
Variable & AIC.diff & AIC.diff.sigma \\ 
  \hline
program\_code & 24382.47 & 16607.68 \\ 
  flag\_id & 23711.29 & 17163.94 \\ 
  HPBCAT2 & 20183.28 & 5765.59 \\ 
  HPBCAT & 20178.74 & 5576.42 \\ 
  yy & 14357.75 & 8612.54 \\ 
  mm & 3725.06 & 2511.55 \\ 
  sharkbait & 994.97 & 454.07 \\ 
   \hline
\end{tabular}
\end{table}

}


\newcommand{\makonorthaic}{
% latex table generated in R 3.2.0 by xtable 1.7-4 package
% Sat Jul 04 14:52:03 2015
\begin{table}[ht]
\centering
\caption{AIC improvement over null model for mako.north from a single explanatory variable} 
\label{mako.north:aic1}
\begin{tabular}{lrr}
  \hline
Variable & AIC.diff & AIC.diff.sigma \\ 
  \hline
HPBCAT2 & 6433.14 & 642.33 \\ 
  HPBCAT & 6417.24 & 640.61 \\ 
  mm & 2845.80 & 216.50 \\ 
  yy & 2216.05 & 401.91 \\ 
  flag\_id & 115.74 & 61.51 \\ 
  program\_code & 114.17 & 44.28 \\ 
  sharkbait & 15.46 & 3.29 \\ 
   \hline
\end{tabular}
\end{table}

}


\newcommand{\makosouthaic}{
% latex table generated in R 3.2.0 by xtable 1.7-4 package
% Sat Jul 04 14:52:03 2015
\begin{table}[ht]
\centering
\caption{AIC improvement over null model for mako.south from a single explanatory variable} 
\label{mako.south:aic1}
\begin{tabular}{lrl}
  \hline
Variable & AIC.diff & AIC.diff.sigma \\ 
  \hline
program\_code & 1693.97 & 251.6 \\ 
  flag\_id & 1438.39 & --- \\ 
  HPBCAT2 & 1202.23 & 79.45 \\ 
  HPBCAT & 1127.42 & 34.67 \\ 
  yy & 875.66 & 389.5 \\ 
  mm & 733.11 & 178.28 \\ 
  sharkbait & 8.37 & -1.89 \\ 
   \hline
\end{tabular}
\end{table}

}


\newcommand{\ocsaic}{
% latex table generated in R 3.2.0 by xtable 1.7-4 package
% Sat Jul 04 14:52:03 2015
\begin{table}[ht]
\centering
\caption{AIC improvement over null model for ocs from a single explanatory variable} 
\label{ocs:aic1}
\begin{tabular}{lrr}
  \hline
Variable & AIC.diff & AIC.diff.sigma \\ 
  \hline
yy & 3726.21 & 958.91 \\ 
  program\_code & 2781.62 & 1369.09 \\ 
  flag\_id & 1499.89 & 772.13 \\ 
  sharkbait & 887.50 & 103.91 \\ 
  HPBCAT2 & 308.22 & 1656.57 \\ 
  HPBCAT & 256.09 & 1658.57 \\ 
  mm & 143.92 & 185.56 \\ 
   \hline
\end{tabular}
\end{table}

}


\newcommand{\silkyaic}{
% latex table generated in R 3.2.0 by xtable 1.7-4 package
% Sat Jul 04 14:52:03 2015
\begin{table}[ht]
\centering
\caption{AIC improvement over null model for silky from a single explanatory variable} 
\label{silky:aic1}
\begin{tabular}{lrr}
  \hline
Variable & AIC.diff & AIC.diff.sigma \\ 
  \hline
program\_code & 10913.31 & 6886.75 \\ 
  flag\_id & 10144.24 & 5446.47 \\ 
  sharkbait & 5382.46 & 1713.24 \\ 
  yy & 4986.96 & 3273.31 \\ 
  HPBCAT2 & 1639.75 & 4527.65 \\ 
  HPBCAT & 1628.44 & 4513.75 \\ 
  mm & 957.50 & 358.96 \\ 
   \hline
\end{tabular}
\end{table}

}



\subsection{Introduction}

%This paper follows from the previous indicator based analysis presented to the Western and Central Pacific Fisheries Commission (WCPFC) Scientific Committee (SC7, Clarke et al. 2011), stock assements (Rice et al. 2014, Rice et al.2013, Rice et al.2012) \hllb{(cite the standardization papers, cite ISC work?)}. The developments presented here include additional analyses of the Secretariat of the Pacific (SPC) data holdings for silky caught in longline and purse seine fisheries in the Western and Central Pacific Ocean (WCPO), though we note that some previous data (Japan) was not available for this effort. Standardized catch per unit of effort (CPUE) series are developed for the main shark species.  
\emph{Note: The current analysis does not construct inputs to use for stock assessments or catch estimates. Our goal is to highlight general trends in population abundance over time, to be interpreted together with other indicators as outlined above. We recommend that catch rates standardization for stock assessments or catch estimates be conducted independently.}


% intro      
Catch-per-unit-effort data (CPUE) are commonly used as an index of abundance for marine species. However, multiple factors---fishing technique, season, bait type, etc.---can alter the relationship between CPUE and abundance, especially in complex fisheries systems comprising of several fleets and spanning large spatial and temporal scales. Nominal catch rates must thus be standardized to account for changes in the relative prevalence of these factors over time. This is typically done \via the use of models in the GLM family, which allows us to model the relationship of CPUE \vs a set of explanatory variables to be standardized against, but these variables must be defined for each observation. The dataset used in the current analysis provides many such candidate variables, but, given the diversity of observer programs represented, few had enough coverage to be retained in the final models. The available variables are described in Table \ref{tbl:glm-vars} (see also Table 2 in \citealt{Francis2014_a} for an overview of the use of variables in shark CPUE standardizations).

% issue with sharks cpue standrz
CPUE data for species such as sharks often have a large proportion of observations (sets) with zero catch, while at the same time also including instances of large catches (`long tails'). These uncommon instances of high catches can occur when areas of high shark densities are accidentally encountered, but also when fishing vessels engage in anecdotal shark targeting behaviour. The co-existence of both high proportions of zero catches \emph{and} long tails resuls in over-dispersed data, and is typical of bycatch species \citep{Ward2005_a}. These features are challenging to account for from a statistical point of view, and have been reviewed at length in the literature on by-catch analyses (Bigelow et al.  2002; Campbell  2004, Ward and Myers  2005; Minami et al. 2007).

Error distributions for by-catch species have been discussed at length in previous publications as these data are notoriously hard to model properly due to the high proportion of zeroes \citep{...}. We achieved significant improvements in model diagnostics by allowing multiple parameters in the error distribution to be fit. This is because accounting for the large amount of zeroes in shark CPUE catch data often comes at the expense of modelling large catch events, since the dispersion parameter which controls the length of the tail is assumed to be constant over all factors. This is especially a problem when the mean of the distribution is close to zero or one, as in those instances the probability of getting large events if mostly controlled by the dispersion parameter (unlike when the mean is larger and the tail is not as pronounced). However, whenever conditions are good for sharks or targeting takes place, larger catch events can happen and not modelling them properly means we are missing important drivers. Typically, this can seen as a bump in the right-hand side of qqnorm plots.
This approach is similar in spirit to the zero-inflated-neg-bin that has been advocated by multiple authors (brodziak) but is less computationally intensive.
The main advantage of the zero inflated approach is that these techniques can model the overdispersion in both the zeros and the counts as opposed to just the counts (negative binomial) and deal with overdispersion better than other models (such as the quasi-Poisson). A drawback of the zero inflated approach is that it is data intensive and the models often fail to converge.

Further, catch data for non-target species (and sharks in particular) often contain large numbers of observed zeros as well as large catch values which need to be explicitly modelled 
More on data issues: Recent work by \citet{Clarke2011_a} noted gaps in observer data in terms of time and  space continuity, reporting rate, and identification with respect to sharks.

% talk about stocks and year span 
Standardized CPUE series for the longline bycatch fisheries were developed using generalized linear models. We did not generate indices from purse-seine observer data. The number of hooks in a longline set was used as a measure of effort.
See also Clarke et al., (2011, 2011b),  Walsh and Clarke (2011), Rice and Harley (2013) for past work on shark CPUE standardization in the Western Central Pacific.
\subsection{Methods}
\label{cpuemeth:datafilter}
\subsubsection{Stock definition for the purpose of the analysis}

Silky and oceanic white tip sharks are observed mainly in the equatorial waters in the purse seine fishery (Figure \ref{...}), and from about -25\degree~S to 25\degree~N in the longline fishery (Figure 1). Silky and oceanic white tip sharks have been assessed \citep{Rice2012_a, Rice2013_a} as a single stock in the WCPO, and are presented in this analysis as a single stock.  Thresher, mako and blue sharks occur more frequently in cold, temperate waters, and generally believed to be separated into northern and southern stocks. For instance, blue sharks in the North Pacific have been subject to multiple stock assessments as a single stock \citep{XXXX}. These temperate species will thus be analysed as individual stocks. Porbeagle sharks are only found in the southern hemisphere and will also be analysed as a single stock. Hammerheads 

To further define the expected geographic range, we defined a coarse climate `envelope' based on sea surface temperature. This aid in distinguishing between zero catches in areas where the species does not occur from zero catches in areas where the species occurs but was not caught. Temperature data were downloaded from the GODAS database \citep{GODASXXXX_a} and matched to the observer data on a set by set basis. The temperature range of a species was defined as the minimum and maximum of the monthly mean sea surface temperatures (SST) of cells with positive catches for that species (see Table \ref{meth:temprange}). SST was measured as the temperature predicted by GODAS at the 5 meters depth. Only cells for which all mean monthly temperatures fell within this range were retained. %\hllb{REWORK. Note that this is a minimal filter and it was only used to exclude the most improbable cells from which a species could be seen.}
\hl{Looked at range of SST where positive catches occured, selected cells where median falls within this range.}
\subsubsection{Data trimming}

Data were cleaned following the general method outlined in Appendix section \ref{app:datacleaning}. Records from the US observer programs (Hawaii and American Samoa) were excluded from the analysis as they were only available up to 2011. Records from the Papua New Guinea observer program were removed as vessels in the fleet frequently target sharks. We also removed records from any observer programs for which we had less than 100 sets. Extreme catch events greater than the $97.5^{th}$ quantile were also removed. Finally, year effects were only estimated if there were at least 50 sets observed in that year.

Although a much smaller proportion of the overall dataset (6.5\% of the sets), the targeting sets represent significant shark catch (82\% of the total silky shark catch). Therefore the dataset was examined with respect to variables relating to whether sharks were the intentional target of the set. 

Shark targeting sets were deemed to be sets where the observer had marked that the set was intentionally targeting sharks of any species, \hlam{whether shark bait was used, or whether shark lines were used}. We also removed the data from the PGOB because of frequent shark targeting. 

The results of these filtering rules are listed by species in Table \ref{XXX}.

\subsection{Additional categorical variables}
1. Day category
2. HPBCAT
%Because silky sharks are an epi-pelagic tropical species, all sets that occurred in water colder than 25\%??C were discarded, this left 95\% of the sets with a non-zero catch (Figure 2). The effect of hooks between floats (a proxy for depth) was investigated independently and sets with greater than 30 hooks between floats were discarded, this left 90\% of the sets with a non-zero catch (Figure 2). National affiliation of the fishing vessel was included in the data set, and only those nations that had greater than 100 sets since 1995 were used. The last variable that resulted in a culling of the data set was that based on non-zero CPUE for unidentified sets (sets where the target is marked as unidentified) as a function of national affiliation. Flagged vessels where the average positive CPUE was 3 times larger than the mean CPUE for all other nations combined were removed from the bycatch longline data under the premise that these vessels were targeting sharks.

%Latitude and longitude were truncated to the nearest 1\degree; this location information was used to calculate the set specific association with a 5\degree square (referred to hereinafter as cell). Date of set was used to calculate the year, month, quarter and trimester of the set. Set time was used to calculate the time category of the day in sixths starting at midnight. A non-target data set was created as a result of filtering data sets according to the above rules as well as filtering sets where sharks were the intentional target. This was done under the premise that the factors leading to non-zero catch rates when targeting sharks would be different than factors that lead to non-zero catch rates when not targeting sharks. 


\subsection{Overview of GLM Analyses}
\subsubsection{Notes on error distributions:} 

The \hl{ filtered datasets} were standardized using generalized linear models (McCullagh and Nelder 1989) using the software package R (www.r-project.org). Multiple assumed error structures were tested including the delta lognormal approach (DLN) (Lo et al. 1992, Dick 2006, Stefansson 1996, Hoyle and Maunder 2006), zero-inflated poisson and negative binomial models, the tweedie distribution (cite), and negative binomial models with mu and sigma modelled. Due to its superior performance both in run time and model diagnostics, we retained the latter and only present those results here.
	   The negative binomial (Lawless 1987): is typically more robust to issues of overdispersion (overdispersion can arise due to excess zeros, clustering of observations, or from correlations between observations) was also used. This model has been advocated as a model that is more robust to overdispersion than the Poisson distribution (McCullagh and Nelder 1991), and is appropriate for count data (Ward and Myers  2005), but does not expressly relate covariates to the occurrence of excess zeros (Minami et al. 2007).

%  		Mixture models such the zero inflated Poisson (ZIP) and zero inflated negative binomial (ZINB) (Zuur 2009, Cunningham and Lindenmayer 2005, Welsh et al. 2000): these models are useful for modelling counts of rare species when the number of zero observations is larger than expected. Zero inflated models are a process similar to the delta approach in which the presence or absence of the catch is modelled orthogonally to the size of the catch (Welsh et al 2000), however unlike the delta approach the count data can include zeros. These zeros could result from predator satiation, competition for hooks, or disinterest (called true zeros) as opposed to design errors, sampling errors, observer errors or zeros resulting from sampling outside the habitat range (called false zeros). The total probability of a zero count is then,
\subsubsection{Procedure for model selection}

%\subsubsection{Note on interactions between year and observer program}

% residuals: qqnorm overall summary and residuals vs. explanatory factors


Because flag and observer programs are highly correlated, we used
observer program as an explanatory categorical variable as it tended
to explain a higher proportion of the data when used on its own than
flag (except in the case of XXXX and XXXX where flag explained the most). We also explored adding an interaction between year and observer
program, as for some species of less mobile sharks we could expect to
see local trends in annual abundance that are reflected in the
observer program data. We checked for the relevance of including
interactions early in the model selection process \hl{but did not proceeded with
an interaction for the remaining of the model selection}.

%as doing so would entail if the AIC
%score when interactions are allowed is at least 50 lower than with
%additive effects only.



\subsubsection{Calculation of year indices and confidence intervals}
Year effects could be extracted as is from the model output as there were no interactions between year and other variables, and year was not included as an explanatory variable for the $\sigma$ models. Confidence intervals were computed with the function \texttt{confint} in \texttt{gamlss}.
%Multiple methods of calculating the indices of abundance and confidence intervals exist depending on the model type (Shono H. 2008, Maunder and Punt 2004). In this study estimates were %calculated by predicting results based on the fitted model and a training data set that included each year effect and  the mean effect for each covariate (Zuur et al 2009). Confidence %intervals were calculated as ?1.96* SE, where SE is the standard error associated with the predicted year effect term.  Appendices hold the model diagnostics.


\subsection{Results}
%----------------------------------------------------------------------------------------
\clearpage

 
\begin{description}
\item[Blue shark (\emph{Prionace glauca}), north Pacific] Both the standardized and nominal CPUEs of blue shark in the north Pacific show a declining trend starting in 1999 and 1998, respectively. Data points for 2011 and 2012 are unavailable due to low sample size.  
 
 \item[Blue shark (\emph{Prionace glauca}), south Pacific]  Both the standardized and nominal CPUEs for blue shark in the south Pacific show declines in the initial 1995-2003 and late 2010-2015 periods, with relatively stable CPUEs between 2004  and 2009.  
 
 \item[Mako shark (\emph{Isurus oxyrinchus} and \emph{Isurus paucus}) in the north Pacific] The standardized and nominal CPUEs share the same trajectories (Figure \ref{nmak:cpue}), but on a slightly different scale for the first 6 years (1995-2001).  The largest difference in the nominal and standardized CPUE is in the final year, where the standardized CPUE declines sharply in contrast to the nominal, but years 2011 and 2012 were excluded from the standardization due to poor sample sizes (Figure \ref{nmak:cpuevars}).
 
\item[Mako shark (\emph{Isurus oxyrinchus} and \emph{Isurus paucus}) in the south Pacific] The standardized CPUEs show a more stable trend in relative abundance than the nominal CPUEs, although both have low points in 2002 and 2014. In addition, the standardized CPUE peaks in 2010, whereas the nominal is the highest in 1996.  
 
\item[Oceanic whitetip shark (\emph{Carcharhinus longimanus})] The standardized oceanic whitetip shark trend decreases steadily over 1995-2014.  The standardized trend shows a slightly steeper decline than the nominal, with the most noticeable departure from the nominal being the large decrease from 2013-2014 in the standardized CPUE.%   Diagnostics indicate no departure from the normality assumptions made in the mode.
 
 \item[Silky shark (\emph{Carcharhinus falciformis})] Standardized silky shark trends in the WCPO showed high inter-annual variability with an initial decline from 1995-2000 followed by a slight increase until 2010, followed by a steep decline. This mirrors the trends seen in the nominal CPUE, albeit with a lesser variability.
 
\item[ Thresher sharks (\emph{ Alopias superciliousus, vulpinus, \& pelagicus})]  Standardized CPUE values for thresher sharks were similar to the nominal CPUE except for additional variability in the nominals. They both rise for the first 6 years of the series (1995-2001) but diverge afterwards. For the years 2002-2014, the standardized CPUE is less variable showing a slightly decreasing CPUE from 2003-2011. The last three years of both the standardized and nominal CPUEs show a steep decline.  The CPUE from the thresher complex (bigeye, common and pelagic) is difficult to interpret as the second most commonly reported thresher species is the general ``thresher shark" category. 
 
\item[Hammerhead sharks (\emph{ Sphyrna mokarran, lewini, zygaena, \& Eusphyra blochii})]  Standardized CPUE for the hammerhead complex shows a large increase from the 3rd to the 6th year of the study period (1997-2001), with a relatively stable CPUE thereafter (2002-2013, regions 3 and 5 in the longline database). Similar to the thresher shark complex, the CPUE series representing the hammerhead complex are difficult to interpret because more than half of the observations in the study period (1995-2014) were made  to a generic ``hammerhead'' category.
 
 \item[Porbeagle shark (\emph{ Lamna nasus}) ] The standardized CPUE for porbeagle shark was close quite similar to the nominal CPUE, showing an increase in the first three years of the time series, followed by a decline from 1999 to 2003, and a monotonic increase thereafter.
 
 
%\item[ Whale Shark (\emph{ Rhincodon typus}) ] Whale shark interactions are generally reported only by the purse seine fishery. These observations are subject to considerable spatial and temporal heterogeneity and likely to have been affected by changes in observer coverage and reporting practices in recent years. The fate of whale sharks following interactions is also uncertain and information on key biological processes are limited. Given the current SPC data holdings only limited analysis for whale sharks in the WCPO is considered to be feasible.
 
 
\end{description}

%\addcenterfig[Blue shark CPUE indicators. Proportion of positive sets, observer data.]{fig:bshcp1}{FIG_xx_pcntpos_reg_BSH}
%\addcenterfig[Blue shark CPUE indicators. Nominal CPUE, sharks per 1000 hooks, observer data.]{fig:bshcp2}{FIG_xx_nomCPUE_reg_BSH}
%\addcenterfig[Blue shark CPUE indicators. Standardized blue shark CPUE based on the negative binomial model for observer data in the northern hemisphere.]{fig:bshcp3}{LLcpue_BSH_north_NB_step}
%\addcenterfig[Blue shark CPUE indicators. Standardized CPUE, zero inflated negative binomial Southern Hemisphere, observer data.]{fig:bshcp4}{ll_cpue_BSHzinb_nominal}

%----------------------------------------------------------------------------------------
          
%\addcenterfig[Mako shark CPUE indicators. Proportion of positive sets, observer data.]{fig:makcp1}{FIG_xx_pcntpos_reg_MAK}
%\addcenterfig[Mako shark CPUE indicators. Nominal CPUE, sharks per 1000 hooks, observer data.]{fig:makcp2}{FIG_xx_nomCPUE_reg_MAK}

%\addcenterfig[Mako shark CPUE indicators. Standardized CPUE, mako shark in the northern hemisphere.]{fig:makcp3}{LLcpue_MAKO_north_NB_cpue}
%\addcenterfig[Mako shark CPUE indicators. Standardized CPUE, mako shark in the southern hemisphere.]{fig:makcp4}{LLcpue_MAKO_south_NB_cpue}

%----------------------------------------------------------------------------------------

%\addcenterfig[Silky shark CPUE indicators. Proportion of positive sets, observer data.]{fig:falcp1}{FIG_xx_pcntpos_reg_FAL}
%\addcenterfig[Silky shark CPUE indicators. Nominal CPUE, sharks per 1000 hooks, observer data.]{fig:falcp2}{FIG_xx_nomCPUE_reg_FAL}

%\addcenterfig[Silky shark CPUE indicators. Standardized CPUE from longline observer data for silky sharks.]{fig:falcp3}{LLcpue_FAL_NB_cpue}

%----------------------------------------------------------------------------------------

%\addcenterfig[Oceanic whitetip shark CPUE indicators. Proportion of positive sets, observer data.]{fig:ocscp1}{FIG_xx_pcntpos_reg_OCS}
%\addcenterfig[Oceanic whitetip shark CPUE indicators. Nominal CPUE, sharks per 1000 hooks, observer data.]{fig:ocscp2}{FIG_xx_nomCPUE_reg_OCS}

%\addcenterfig[Oceanic whitetip shark CPUE indicators. Standardized CPUE based on negative  binomial models applied to observer data.]{fig:ocscp3}{LLcpue_OCS_NB_cpue}

%----------------------------------------------------------------------------------------
%\subsubsection{Thresher Shark}
          
%\addcenterfig[Thresher shark CPUE indicators. Proportion of positive sets, observer data.]{fig:thrcp1}{FIG_xx_pcntpos_reg_THR}
%\addcenterfig[Thresher shark CPUE indicators. Nominal CPUE, sharks per 1000 hooks, observer data.]{fig:thrcp2}{FIG_xx_nomCPUE_reg_THR}
%\addcenterfig[Thresher shark CPUE indicators.  Standardized CPUE of thresher shark based on longline observer data.]{fig:thrcp3}{LLcpue_THRESHER_NB_cpue}

%----------------------------------------------------------------------------------------          
\subsection{Conclusions}            


      
\clearpage     
      
\begin{table}[!h]
\begin{center}
\caption{Summary of temperature ranges by species used as filters for cells to retain in the CPUE analysis. \label{meth:temprange}}
\begin{tabular}{l|c|c}
Species & Minimum T(\degree C) & Maximum T(\degree C)\\
\hline
\hline
Blue shark& 10 & 30\\
Hammerhead sharks& 13 & 30\\
Mako sharks& 11 & 30\\
Oceanic whitetip shark&18&30\\
Porbeagle shark&10&26\\
Silky shark&18&30\\
Thresher sharks&11&30\\
%Using range of 10 to 30 for BSH
%Using range of 18 to 30 for FAL
%Using range of 13 to 30 for HHD
%Using range of 11 to 30 for MAK
%Using range of 18 to 30 for OCS
%Using range of 10 to 26 for POR
%Using range of 11 to 30 for THR      
\end{tabular}
\end{center}
\end{table}
% include hbf by program_code
% dot map of catches over time
% include table of AIC change by variables on their own


\begin{table}[!h]
\label{tbl:glm-vars}
\begin{center}
\begin{tabular}{l|l|p{7cm}|c}
Variable name & Symbol & Explanation & \% records present\\
\hline
\hline
Year & $\beta_Y$ & Required to estimate year effect & 100\\
Month & $\beta_M$ & Captures seasonal variability & 100\\
Observer program & $\beta_O$ & Country hosting the observer program & 100\\
Vessel flag & $\beta_F$ & Note: correlated with observer program & 100\\
Hooks-beween-floats& $\beta_{HBF}$ & Indicator of catchability for surface-dwelling species\\
Shark bait &&\\
%Number of shark lines&&\\
%Lighsticks &&\\
Shark target&&Sharks explicitly defined as targets?\\
SST & $SST$ & 100\\
Day category && Day or night, before or after sunrise/sunset?&100\\

\end{tabular}
\end{center}
\end{table}

\subsection{Model structure by species}
in appendix: summarize proportion of data removed for each species (temperature, observer, quantile), e.g. North Mako: removing HW removes a lot of data!! check for catches...

\subsection{Model diagnostics}
Used quantile residuals. 

\subsection{Conclusions}
The signals from the nominal  CPUE data can be heavily influenced by factors other than abundance and therefore a procedure to standardize CPUE data for changes in factors  that do not reflect changes in abundance is usually recommended. 

\clearpage
\begin{landscape}
\thispagestyle{empty}
 \begin{table}[!h]
\caption{Summary of number of records removed by filter type for each species before GLM analyses. HW/AS and PG refer to the Hawaai/American Samoa and the Papua New Guinea observer programs. OB sampling refers to records removed from observer programs with few records. See summary in section \ref{cpuemeth:datafilter} \label{tbl:glmdata}}
\begin{center}
\begin{tabular}{l|c|c|c|c|c|c|c|c}
Species & Hemisphere & SST range & max quantile & HW/AS & PG & OB sampling & \# rows left\\
\hline
\hline
Blue shark, south&41276&1234&309&3449&571&21&19660\\ 
Blue shark, north&25244&0&805&36818&0&35&3618\\ 
Hammerhead sharks&0&4999&12&41072&571&21&19845\\ 
Mako sharks, south&41276&1419&130&3359&571&21&19744\\ 
Mako sharks, north&25244&0&97&37536&0&35&3608\\ 
Oceanic whitetip shark&0&10266&171&38532&570&21&16960\\ 
Porbeagle shark&41276&18038&78&0&0&122&7006\\ 
Silky shark&0&10266&127&38563&563&21&16980\\ 
Thresher sharks&0&1419&290&40860&571&21&23359\\

 
\end{tabular}
\end{center}
\end{table}
%%%%%%%%%%%%%%%%%%%%%%%%%%%%%%%%%%%%%%%%%%%%%%%
\begin{table}[!h]
\caption{Summary of model structures retained for CPUE standardization of each species}
\begin{center}
\begin{tabular}{l|c|c|c}
Species & Model $\mu$& model $\sigma$ & \% deviance\\
\hline
\hline
Blue shark, northern stock  & year + program + HPBCAT2 + month + sharkbait & program + HPBCAT2 + month&\\
Blue shark, southern stock & year + flag + HPBCAT2 + month + sharkbait & \\ 
Mako, southern stock & year + program + HPBCAT2 + month + sharkbait & program + month & ---\\
Mako, northern stock & year + program + month + sharkbait & HPBCAT2 \\
Oceanic white tip & year + program + HBPCAT2 + month + sharkbait & program \\
Thresher sharks& year + program + HPBCAT2 + month & program + HBPCAT2 \\
Hammerheads& year + program + HPBCAT2 + month + sharkbait & sharkbait \\
Oceanic whitetip shark& year + program +HBPCAT2 + month + sharkbait & program\\
Silky shark & program + year + HPBCAT2 + month + sharkbait & program + sharkbait + month\\
Porbeagle & year + flag + HPBCAT2 + month & flag + month\\ 

\end{tabular}
\end{center}
\end{table}
\end{landscape}


\BSHnorthaic
\BSHsouthaic
\FALaic
\HHDaic
\MAKnorthaic
\MAKsouthaic
\OCSaic
\PORaic
\THRaic
\end{document}
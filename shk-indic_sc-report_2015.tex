%%%%%%%%%%%%%%%%%%%%%%%%%%%%%%%%%%%%%%%%%
% Journal Article
% SHARK INDICATORS IN THE WESTERN CENTRAL PACIFIC
% June 2015
%
% Author: Joel Rice (joelrice@uw.edu )
%
%
%%%%%%%%%%%%%%%%%%%%%%%%%%%%%%%%%%%%%%%%%

%----------------------------------------------------------------------------------------
%  PACKAGES AND OTHER DOCUMENT CONFIGURATIONS
%----------------------------------------------------------------------------------------



\documentclass[12pt]{article}

\input{input_package.tex}


%\graphicspath{ {C:/Users/sheltonh/Dropbox/SHK-indicators-2015/GRAPHICS/} }
\graphicspath{ {C:/Projects/SHK-indicators-2015/GRAPHICS/} }

\usepackage[english]{babel}

\input{WCPFC-title-page.tex}
\reportauthor{Joel Rice\footnote{Joel Rice Consulting Ltd.}, Laura Tremblay-Boyer, and Shelton Harley}
\reporttitle{Analysis of stock status and related indicators for key shark species of the Western Central Pacific Fisheries Commission}
\reportnumber{SA-IP-05}



%----------------------------------------------------------------------------------------
%  Document content
%----------------------------------------------------------------------------------------


\begin{document}

\wcpfctitlepage

\section*{Executive Summary}


\section{Introduction}
The status of the many  shark species, espically those designated as  (\emph{key shark species}) in the western and central Pacific Ocean was is under review and instead of doing a southern blue shark stock assessment you all asked for this. An indicator analysis of blue, mako, thresher, silky and oceanic white tip sharks in the waters of the WCPO. 

  We didn't do any fancy assessment work or models, but rather make colorul plots and tabulate laregly useless statistics ( the geometric mean of the sandardized counts of other sharks has decreased relative to the base year but is comparable to the initial year).  All in all this paper should give you a good understanding of the uncertainty regarding any species population viability, and even more confusioin regarding what we can, have or would do about it.  Becauses sharks are often caught as bycatch in the Pacific tuna fisheries (though some directed mixed species fisheries, sharks and tunas/billfish, do exist) sharks are doomed.
  
  While we cannot specify a percent  reduction in fishing mortality  of approximately  needed for any specific  species to reach  MSY levels in the western central Pacific Ocean, we do know that-  based on modeling of the factors influenciing the catch rate - the most effective way to improve population outlook   would be the banning of shark lines.

This paper presents an analysis of Secretariat of the Pacific Community - Oceanic Fisheries Programme (SPC-OFP) data holdings for sharks taken in longline and purse seine fisheries in the Western and Central Pacific Ocean (WCPO). The framework for the analysis is a series of indicators of fishing pressure and stock status that were first described in the Shark Research Plan presented to the sixth meeting of the Western and Central Pacific Fisheries Commission's (WCPFC) Scientific Committee (SC6; Clarke and Harley 2010). A preliminary indicator-based analysis of SPC data holdings was presented to the Commission in December 2010 (Clarke et al. 2010) with an extinsive review of the fisheries and data sources presented to SC7 (Clarke et al. 2011).

\section{General Methods}
        \subsection{Description of Data Sources}
         
\addcenterfig[Map of WCPO and regions used for the analysis.]{fig:regions}{FIG_1_MAP}
\addcenterfig[Map of WCPO and observed effort and observed shark catch. ]{fig:sets}{FIG_2_MAP_sets}
\addcenterfig[Observed purse seine in the  WCPO showing the top four fishing nations and all others combined.  ]{fig:pssets}{FIG_xx_PS_eMAP_sets}
 
\clearpage         
        \subsection{Data formatting}
        \subsection{Limitations - Caveats}
        
        
%----------------------------------------------------------------------------------------
%  Distribuion Indicators
%----------------------------------------------------------------------------------------
             
        
\section{Distribution Indicator Analyses}
      \subsection{Introduction}
      \subsection{Methods}
      \subsection{Results}
          \subsubsection{Fishing Effort}
              
\addcenterfig[Aggregate effort by region. \hl{needs updating}]{fig:aggeff}{FIG_xx_agg_eff}
\addcenterfig[Observed effort by region.]{fig:obseff}{FIG_xx_OBS_eff}
\addcenterfig[Logsheet effort by month.]{fig:logeffmnth}{FIG_xx_LOGSHEET_mm}
\addcenterfig[Observed effort by month.]{fig:obseffmnth}{FIG_xx_obsBY_mm}
\addcenterfig[Absolute percent difference in effort between reported (logsheet)  effort and observed effort.]{fig:effdiff}{FIG_xx_obsDIFFlog_mm}
\clearpage         
%----------------------------------------------------------------------------------------
          \subsubsection{Blue Shark}

\addcenterfig[Blue shark distribution indicators. Proportion of positive sets, observer data.]{fig:bsh1}{FIG_xx_pcntpos_reg_BSH}
\addcenterfig[Blue shark distribution indicators. Proportion of 5 degree squares having CPUE greater than 1 per 1000 hooks region, observer data.]{fig:bsh2}{FIG_xx_HIGH_CPUE_BSH}


%----------------------------------------------------------------------------------------
\subsubsection{Mako Shark}

\addcenterfig[Mako shark distribution indicators. Proportion of positive sets, observer data.]{fig:mak1}{FIG_xx_pcntpos_reg_MAK}          
\addcenterfig[Mako shark distribution indicators. Proportion of 5 degree squares having CPUE greater than 1 per 1000 hooks region, observer data.]{fig:mak2}{FIG_xx_HIGH_CPUE_MAK}
          
          
%----------------------------------------------------------------------------------------
 \subsubsection{Silky Shark}
 
\addcenterfig[Mako shark distribution indicators. Proportion of positive sets, observer data.]{fig:fal1}{FIG_xx_pcntpos_reg_FAL}           
\addcenterfig[Silky shark distribution indicators. Proportion of 5 degree squares having CPUE greater than 1 per 1000 hooks region, observer data.]{fig:fal2}{FIG_xx_HIGH_CPUE_FAL}

%----------------------------------------------------------------------------------------
 \subsubsection{Oceanic Whitetip Shark}
 \addcenterfig[Oceanic whitetip shark distribution indicators. Proportion of positive sets, observer data.]{fig:ocs1}{FIG_xx_pcntpos_reg_OCS}          
 \addcenterfig[Oceanic whitetip shark distribution indicators. Proportion of 5 degree squares having CPUE greater than 1 per 1000 hooks region, observer data.]{fig:ocs2}{FIG_xx_HIGH_CPUE_OCS}

%----------------------------------------------------------------------------------------
 \subsubsection{Thresher Shark}
\addcenterfig[Thresher shark distribution indicators. Proportion of positive sets, observer data.]{fig:mak1}{FIG_xx_pcntpos_reg_THR} 
\addcenterfig[Thresher shark distribution indicators. Proportion of 5 degree squares having CPUE greater than 1 per 1000 hooks region, observer data.]{fig:thr2}{FIG_xx_HIGH_CPUE_THR}
          
          
  \subsection{Conclusions}
%----------------------------------------------------------------------------------------
%  Species COmposition
%----------------------------------------------------------------------------------------
      
      
\section{Observed Species Composition Indicator Analyses}
      \subsection{Introduction}
      \subsection{Methods}
      \subsection{Results}
      
  \subsubsection*{Longline}      
  \addcenterfig[Catch Composition Indicators. Sharks Per. 1000 hooks by region, observer data.]{fig:catchcomp1}{FIG_xx_shksP1000Hooks}
  \addcenterfig[Catch Composition Indicators. Sharks Per. 1000 hooks by region, deep sets observer data.]{fig:catchcomp2}{FIG_xx_shksP1000Hooks_deep}  
  \addcenterfig[Catch Composition Indicators. Sharks Per. 1000 hooks by region, deep sets observer data.]{fig:catchcomp3}{FIG_xx_shksP1000Hooks_shallow} 
%------------------------------------------------------------------------------------------
 \subsubsection*{Purse Seine}        
 \addcenterfig[Catch Composition Indicators. Sharks per set, observer data.]{fig:catchcomp4}{FIG_xx_PS_shks_set}
 \addcenterfig[Catch Composition Indicators. Sharks per set, associated sets, observer data]{fig:catchcomp5}{FIG_xx_PS_shks_UNAS}  
 \addcenterfig[Catch Composition Indicators. Sharks per set, unassociated sets, observer data.]{fig:catchcomp6}{FIG_xx_PS_shks_ASSO} 

  \addcenterfig[Catch Composition Indicators. Catch composition by proportion , observer data.]{fig:catchcomp7}{catchcomp_xx_PS_comp_reg}
 \addcenterfig[Catch Composition Indicators. Catch composition by proportion, associated sets, observer data]{fig:catchcomp8}{catchcomp_xx_PS_comp_reg_UNAS}  
 \addcenterfig[Catch Composition Indicators. Catch composition by proportion, unassociated sets, observer data.]{fig:catchcomp9}{catchcomp_xx_PS_comp_reg_ASSOC} 

 
 
 \subsection{Conclusions}
      
      
 \clearpage          

%----------------------------------------------------------------------------------------
%  CPUE INDICATORS
%----------------------------------------------------------------------------------------
\section{Catch Per Unit Effort indicator analyses}
      \subsection{Introduction}
      This paper follows from the previous indicator based analysis presented to the Western and Central Pacific Fisheries Commission (WCPFC) Scientific Committee (SC7, Clarke et al. 2011), stock assements (Rice et al. 2014, Rice et al.2013, Rice et al.2012)   (cite the standardization papers cite ISC work?). The developments presented here include additional analyses of the Secretariat of the Pacific (SPC) data holdings for silky caught in longline and purse seine fisheries in the Western and Central Pacific Ocean (WCPO), though we note that some previous data (Japan) was not available for this effeort. Standardized catch per unit of effort (CPUE) series are developed for the main shark species.  
      
The framework for the analysis is not  to construct inputs for stock assessment or estimate catch, it is designed to illustrate general population trends via   catch rate. It is recommended that infrence to develop catch estimates or other stock assessment inputs be conducted independently. The SPC longline observer database contains records from 1985 to recent years, however silky sharks were not routinely identified to species until 1995, hence the dataset used in this analysis spans the years 1995-2014. Recent work by Clarke et al. (2011) noted gaps in observer data in terms of time and  space continuity, reporting rate, and identification with respect to sharks. Silky and oceanic white tip sharks are observed mainly in the equatorial waters in the purse seine fishery (Figure 1), and from about -25??S to 25??N in the longline fishery (Figure 1). Silky and oceanic white tip sharks have been assessed (Rice et al 2012, Rice et al 2013) as a single stock in the WCPO, and are presented in this analysis ass one stock (not regionally).  Thresher, mako and blue sharks are more common in cold and temperate waters, and generally believed to constitute two seperate stocks, in the north and south. Blue shark in the north pacific have been subject to multiple stock assessments as a single stock.  These temperate species stocks will be presented as individual  stocks. 

CPUE data for species such as sharks often have a large proportion of observations (or sets) with no catch, and also include observations with large catches when areas of higher densities are encountered; this is typical of bycatch species (Ward and Myers  2005). The signals from the nominal  CPUE data can be heavily influenced by factors other than abundance and therefore a procedure to standardize CPUE data for changes in factors (e.g. fishing technique, season, bait type) that do not reflect changes in abundance is usually recommended. Nominal CPUE data for bycatch can be more variable than expected (i.e., overdispersed) with many outlying data points from uncommonly high catch rates. These outlying data points can sometimes be a function of shark targeting.



      \subsection{Methods}
      This analysis follows the work of Clarke et al., (2011, 2011b),  Walsh and Clarke (2011), Rice and Harley (2013) however the regions for this study differ slightly. Because silky sharks are tropical species this led to the analysis being considered for one region, from 25??S to 25??N and bordered on the east and west by the WCPFC Statistical Area. A comprehensive overview of the observer logsheet data and a characterization of the fisheries in which each species is caught is presented in the preious sections, what follows is a summary of the methods used in this analysis.
      
      The data were validated and trimmed (records with missing values for key explanatory variables removed) to include only relevant data from the species 'core' habitat. This was done to reduce the already excessive number of zeros in the data, i.e. zero catch where you would not reasonably expect to catch silky sharks.
      
      %Environmental data about temperature, salinity, moon phase, and depth of the 27%??C isotherm downloaded from the GODAS database (GODAS 2011) were matched to the observer data  on set by set basis. 
      
Because silky sharks are an epi-pelagic tropical species, all sets that occurred in water colder than 25%??C were discarded, this left 95\% of the sets with a non-zero catch (Figure 2). The effect of hooks between floats (a proxy for depth) was investigated independently and sets with greater than 30 hooks between floats were discarded, this left 90\% of the sets with a non-zero catch (Figure 2). National affiliation of the fishing vessel was included in the data set, and only those nations that had greater than 100 sets since 1995 were used. The last variable that resulted in a culling of the data set was that based on non-zero CPUE for unidentified sets (sets where the target is marked as unidentified) as a function of national affiliation. Flagged vessels where the average positive CPUE was 3 times larger than the mean CPUE for all other nations combined were removed from the bycatch longline data under the premise that these vessels were targeting sharks.

Latitude and longitude were truncated to the nearest 1%??; this location information was used to calculate the set specific association with a 5??square (referred to hereinafter as cell). Date of set was used to calculate the year, month, quarter and trimester of the set. Set time was used to calculate the time category of the day in sixths starting at midnight. A non-target data set was created as a result of filtering data sets according to the above rules as well as filtering sets where sharks were the intentional target. This was done under the premise that the factors leading to non-zero catch rates when targeting sharks would be different than factors that lead to non-zero catch rates when not targeting sharks. 

Although a much smaller proportion of the overall dataset (6.5\% of the sets), the targeting sets represent significant shark catch (82\% of the total silky shark catch). Therefore the dataset was examined with respect to variables relating to whether sharks were the intentional target of the set. Silky shark CPUE was plotted as a function of the variables sharkline, shark bait, shark target against date of set (Figure 3). Inspection of these covariates led to the separation of shark-targeting sets and non-targeting (bycatch) sets. Shark targeting sets were deemed to be sets where the observer had marked that the set was intentionally targeting sharks of any species, whether shark bait was used, or whether shark lines were used. 
The results of these filtering rules are in Table XXX.

\subsection*{Purse Seine data preparation}
The only restriction placed on the purse seine observer data was that the set occurred within the rectangle defined by% 7??N and -12??S Latitude and 139??W to 192??E. The purse seine data was separated into two fisheries, one based on associated sets and one based on unassociated sets.

\section*{CPUE methodology}
CPUE is commonly used as an index of abundance for marine species.  However, it is important that raw nominal catch rates be standardized to remove the effects of factors other than abundance. Further, catch data for non-target species (and sharks in particular) often contain large numbers of observed zeros as well as large catch values which need to be explicitly modelled (Bigelow et al.  2002; Campbell  2004, Ward and Myers  2005; Minami et al.  2007).
Standardized CPUE series for all fisheries (bycatch and target longline; associated and un-associated purse seine fisheries) were developed using generalized linear models. In the longline analyses the number of hooks in a set was the effort measure, whereas for purse seine it was simply the set. It is notoriously difficult to come up with accurate estimates of the true effort that relates to a purse seine set (Punsly, 1987).
\subsection*{POverview of GLM Analyses}
The filtered datasets were standardized using generalized linear models (McCullagh and Nelder 1989) using the software package R (www.r-project.org). Multiple assumed error structures were tested including;
  The delta lognormal approach (DLN) (Lo et al. 1992, Dick 2006, Stefansson 1996, Hoyle and Maunder 2006): this approach is a special case of the more general delta method (Pennington 1996, Ortiz and Arocha 2004), and uses a binomial distribution for the probability w of catch being zero and a probability distribution f(y), where y was log(catch/hooks set), for non-zero catches. An index was estimated for each year, which was the product of the year effects for the two model components,% (1-w)*E(y???y???0). 
		 
	   The negative binomial (Lawless 1987): is typically more robust to issues of overdispersion (overdispersion can arise due to excess zeros, clustering of observations, or from correlations between observations) was also used. This model has been advocated as a model that is more robust to overdispersion than the Poisson distribution (McCullagh and Nelder 1991), and is appropriate for count data (Ward and Myers  2005), but does not expressly relate covariates to the occurrence of excess zeros (Minami et al. 2007).
     
		Mixture models such the zero inflated Poisson (ZIP) and zero inflated negative binomial (ZINB) (Zuur 2009, Cunningham and Lindenmayer 2005, Welsh et al. 2000): these models are useful for modelling counts of rare species when the number of zero observations is larger than expected. Zero inflated models are a process similar to the delta approach in which the presence or absence of the catch is modelled orthogonally to the size of the catch (Welsh et al 2000), however unlike the delta approach the count data can include zeros. These zeros could result from predator satiation, competition for hooks, or disinterest (called true zeros) as opposed to design errors, sampling errors, observer errors or zeros resulting from sampling outside the habitat range (called false zeros). The total probability of a zero count is then,
%Pra(Y_i=0)=Pra(False Zeros)+ (1-Pra(False Zeros) )*Pra(True Zeros)
Therefore, the probability distribution for the zero inflated Poisson is equal to:
%Pra(y_i=0)  =   ??_i+(1-??_i )*e^(???-?????_i )
%Pra(y_i ???|y???_i>0)  =    (1-??_i )*(??^(y_i )*e^(???-?????_i ))/(y_i !) 
Where yi is the size of the catch of the ith set, and distributed % yi ~ Poisson(�i) (�i is the mean of the Poisson distribution), and ??i  is the probability of a false zero. The probability definition for the zero inflated negative binomial is similar,
%Pra(y_i=0)  =   ??_i+(1-??_i )*(k/(??_i+k))^k
%Pra(y_i ???|y???_i>0)  =    (1-??_i )*(??(y_i+k))/(??(k)*??(y_i+1)) ???*(k/(??_i+k))???^k*(1-k/(??_i+k))^(y_i ) 
Where yi is the size of the catch of the ith set, and distributed %yi ~ Negative %Binomial(�i,k), and ??i  is the probability of a false zero. 
Under this parameterization the mean of the negative binomial is ?? and the variance is %(??+??^2)???k. 
The main advantage of the zero inflated approach is that these techniques can model the overdispersion in both the zeros and the counts as opposed to just the counts (negative binomial) and deal with overdispersion better than other models (quasi Poisson).
Each model was fit to the data set independently and  all variables used in the models were  included as categorical factors except the response variables for catch and the effort (silky and SILKYCPUE) and the offset variable (hook\_est); these variables were included in the model as continuous variables (Table 1). Model selection began with regression trees and piecewise ANOVA models for each model (De'ath and Fabricius  2000, Zuur 2009). The Akaike information criterion (AIC) was used as a metric to score the results and determine the final models for each data set.  Model specific fitting criteria and model diagnostics resulted in different variables being chosen for  different data sets and model types. 
Multiple methods of calculating the indices of abundance and confidence intervals exist depending on the model type (Shono H. 2008, Maunder and Punt 2004). In this study estimates were calculated by predicting results based on the fitted model and a training data set that included each year effect and  the mean effect for each covariate (Zuur et al 2009). Confidence intervals were calculated as %�1.96* 
SE, where SE is the standard error associated with the predicted year effect term.  Appendices hold the model diagnostics

      \subsection{Results}
      %----------------------------------------------------------------------------------------
         \clearpage
         \subsubsection{Blue Shark}
           
\addcenterfig[Blue shark CPUE indicators. Proportion of positive sets, observer data.]{fig:bshcp1}{FIG_xx_pcntpos_reg_BSH}
\addcenterfig[Blue shark CPUE indicators. Nominal CPUE, sharks per 1000 hooks, observer data.]{fig:bshcp2}{FIG_xx_nomCPUE_reg_BSH}
\addcenterfig[Blue shark CPUE indicators. Standardized CPUE, zero inflated negative binomial Southern Hemisphere, observer data.]{fig:bshcp3}{ll_cpue_BSHzinb_nominal}

\addcenterfig[Blue shark CPUE indicators. Standardized CPUE (delta lognormal) and nominal CPUE,  Southern Hemisphere, observer data.]{fig:bshcp4}{FIG_xx_LLcpue_BSH_South}




%----------------------------------------------------------------------------------------
  \subsubsection{Mako Shark}
          
\addcenterfig[Mako shark CPUE indicators. Proportion of positive sets, observer data.]{fig:makcp1}{FIG_xx_pcntpos_reg_MAK}
\addcenterfig[Mako shark CPUE indicators. Nominal CPUE, sharks per 1000 hooks, observer data.]{fig:makcp2}{FIG_xx_nomCPUE_reg_MAK}
\clearpage
%----------------------------------------------------------------------------------------
 \subsubsection{Silky Shark}
 
\addcenterfig[Silky shark CPUE indicators. Proportion of positive sets, observer data.]{fig:falcp1}{FIG_xx_pcntpos_reg_FAL}
\addcenterfig[Silky shark CPUE indicators. Nominal CPUE, sharks per 1000 hooks, observer data.]{fig:falcp2}{FIG_xx_nomCPUE_reg_FAL}

%----------------------------------------------------------------------------------------
 \subsubsection{Oceanic Whitetip Shark}
          
\addcenterfig[Oceanic whitetip shark CPUE indicators. Proportion of positive sets, observer data.]{fig:ocscp1}{FIG_xx_pcntpos_reg_OCS}
\addcenterfig[Oceanic whitetip shark CPUE indicators. Nominal CPUE, sharks per 1000 hooks, observer data.]{fig:ocscp2}{FIG_xx_nomCPUE_reg_OCS}

%----------------------------------------------------------------------------------------
  \subsubsection{Thresher Shark}
          
\addcenterfig[Thresher shark CPUE indicators. Proportion of positive sets, observer data.]{fig:thrcp1}{FIG_xx_pcntpos_reg_THR}
\addcenterfig[Thresher shark CPUE indicators. Nominal CPUE, sharks per 1000 hooks, observer data.]{fig:thrcp2}{FIG_xx_nomCPUE_reg_THR}
          
%----------------------------------------------------------------------------------------          
      \subsection{Conclusions}            
      
 \clearpage     
      
      
      
      
     
%----------------------------------------------------------------------------------------
%  Biological
%----------------------------------------------------------------------------------------
      
\section{Biological indicator analyses}
      \subsection{Introduction}
      \subsection{Methods}
      \subsection{Results}
      \subsection{Conclusions}
 \clearpage     
      
%----------------------------------------------------------------------------------------
%  Discussion Type Sections.........
%----------------------------------------------------------------------------------------
         
      
\section{Feasibility of Stock Assessments}
\section{Impact of Recent Shark Management Measures}
 % analysis of fate and condition goes here? 

\section{Recommendations for Future Indicator Work}
\section{Management Implications}

\section*{Acknowledgements}

%----------------------------------------------------------------------------------------
%  REFERENCE LIST
%----------------------------------------------------------------------------------------



\section{Appendices}
% 
\subsection{CPUE Standardizatoin model diagnostics and extra plots}
\addcenterfig[Blue shark CPUE indicators. Standardized CPUE (delta lognormal) model diagnostics binomial component,  Southern Hemisphere, observer data.]{fig:bshcp_AN1}{LLcpue_BSH_South_BIN_DIAG}

\addcenterfig[Blue shark CPUE indicators. Standardized CPUE (delta lognormal), model diagnostics- lognormal  component,  Southern Hemisphere, observer data.]{fig:bshcp_AN2}{LLcpue_BSH_South_logDIAG}

      
      
\end{document}



